\documentclass[12pt]{report}
\usepackage{polyglossia}
\setmainlanguage{vietnamese}
\usepackage{graphicx}
\graphicspath{{images/}}
\usepackage[a4paper]{geometry}
\usepackage{lipsum}
\usepackage{fancyhdr}
\usepackage{indentfirst}
% \setlength{\parindent}{15pt}
\pagestyle{fancy}
\fancyhead{}
\fancyhead[L]{\leftmark}
% \fancyfoot{}
% \fancyfoot[R]{\thepage}
% \fancyfoot[L]{Chương \thechapter}
% \fancyfoot[C]{Bùi Minh Thắng}
\fancyheadoffset[R]{0mm}
\usepackage{fontspec}
\setmainfont{Times New Roman}
\usepackage{tikz}
\usetikzlibrary{calc}
\usepackage{listings}
\usepackage{xcolor}
\usepackage{courier}


\usepackage{biblatex}
\addbibresource{refs.bib}

\title{
    {Thesis Title}\\
    {\large Institution Name}\\
    % {\includegraphics{UET_logo.png}}
}
\author{Bùi Minh Thắng}
\date{March 2025}

\begin{document}
% \fontsize{14pt}{18pt}\selectfont

\newgeometry{
    top=2.2cm,
    bottom=2.2cm,
    left=2.2cm,
    right=2.2cm
}

\begin{titlepage}
    \begin{tikzpicture}[overlay,remember picture]
    \draw [line width=0.5mm]
        ($ (current page.north west) + (2cm,-2cm) $)
        rectangle
        ($ (current page.south east) + (-2cm,2cm) $);
    \draw [line width=0.5m  m]
        ($ (current page.north west) + (1.9cm,-1.9cm) $)
        rectangle
        ($ (current page.south east) + (-1.9cm,1.9cm) $);
    \end{tikzpicture}
    \begin{center}
        \large\textbf{ĐẠI HỌC QUỐC GIA HÀ NỘI}\\
        \large\textbf{TRƯỜNG ĐẠI HỌC CÔNG NGHỆ}\\
        ----------------***---------------

        \vspace{3cm}
        {\large\textbf{BÁO CÁO THU HOẠCH HỌC PHẦN CÁC VẤN ĐỀ HIỆN ĐẠI CỦA TRUYỀN THÔNG VÀ MẠNG MÁY TÍNH}}

        \vspace{3cm}
        {\huge\textbf{Các Khía Cạnh Điều Khiển trong Hệ Thống MEC Hỗ Trợ bởi RIS cho Ứng Dụng Độ Trễ Nghiêm Ngặt}}

        \vfill
        \begin{table}[h]
            \centering
            \large
            \begin{tabular}{ll}
                \textbf{Họ và Tên:} & Bùi Minh Thắng - 23020646 \\
                \textbf{          } & Nguyễn Vũ Minh - 23020629 \\
                \textbf{          } & Ma Đức Minh - 23020626 \\
                \textbf{          } & Nguyễn Hoàng Tùng Dương - 21020182 \\
                \textbf{Người hướng dẫn:} & TS. Nguyễn Ngọc Tân \\
                \textbf{                } & CN. Nguyễn Thái Dương \\
            \end{tabular}
        \end{table}

        \vspace{4cm}
        \textbf{Hà Nội, 2025}
        
    \end{center}
\end{titlepage}

\newgeometry{
    top=2cm,
    bottom=2cm,
    left=3cm,
    right=2cm
}

\chapter*{Lời cam đoan}
Nhóm em xin cam đoan: Báo cáo nghiên cứu khoa học với đề tài “Mô phỏng điều hướng lưu lượng trong O-RAN với xApp dùng Deep Reinforcement Learning” này là của nhóm em. Những gì nhóm em viết ra không có sự sao chép từ các tài liệu, không sử dụng kết quả của người khác mà không trích dẫn cụ thể. Đây là công trình nghiên cứu tập thể nhóm em tự phát triển, không sao chép mã nguồn của người khác. Nếu vi phạm những điều trên, nhóm em xin chấp nhận tất cả những truy cứu về trách nhiệm theo quy định.

\vspace{2cm} % nếu cần tạo khoảng cách từ nội dung phía trên
\noindent
\hfill
\begin{minipage}{0.4\textwidth}
    \centering
    \textbf{Sinh viên}\\[1cm] % khoảng trống cho chữ ký
    Bùi Minh Thắng\\
    Nguyễn Vũ Minh\\
    Ma Đức Minh\\
    Nguyễn Hoàng Tùng Dương\\
\end{minipage}

\chapter*{Lời cảm ơn}
Lời đầu tiên, em xin được gửi lời cảm ơn chân thành tới Khoa Công nghệ Thông tin – Trường Đại học Công nghệ – Đại học Quốc gia Hà Nội đã tạo điều kiện thuận lợi để em được học tập, nghiên cứu và thực hiện đề tài này.

Em xin bày tỏ lòng biết ơn sâu sắc tới thầy Nguyễn Ngọc Tân và thầy Nguyễn Thái Dương đã tận tình hướng dẫn, hỗ trợ em trong suốt quá trình nghiên cứu và triển khai đề tài.

Bên cạnh đó, em xin được bày tỏ lòng biết ơn tới các thầy cô trong khoa đã tận tâm giảng dạy và trang bị cho em những kiến thức quý báu trong suốt quá trình học tập tại trường.

Cuối cùng, em xin chúc các thầy cô, các bạn luôn mạnh khỏe, hạnh phúc và gặt hái nhiều thành công trong cuộc sống.

\chapter*{Tóm tắt}
Báo cáo này trình bày quá trình phát triển và mô phỏng một xApp trong kiến trúc Open RAN, tập trung vào việc điều hướng lưu lượng mạng sử dụng Deep Reinforcement Learning (DRL). Chúng tôi đã xây dựng một xApp để tối ưu hóa việc phân phối lưu lượng mạng trong môi trường O-RAN, sử dụng mô hình DRL để cải thiện hiệu suất mạng. Bằng cách sử dụng mô phỏng ns-3, chúng tôi đã kiểm tra hiệu quả của xApp trong các tình huống thực tế, cho thấy khả năng cải thiện đáng kể trong việc quản lý lưu lượng và tối ưu hóa tài nguyên mạng. Báo cáo cũng cung cấp hướng dẫn chi tiết về cách phát triển xApp và tích hợp nó vào hệ thống O-RAN.

\textbf{Từ khoá:} Open RAN architecture, O-RAN RIC, xApp development, O-RAN use cases, ns-3 O-RAN simulation, DRL for traffic steering, RAN Intelligent Controller.

\tableofcontents

% % \listoffigures

% % \listoftables

% % \chapter*{Thuật ngữ}


\chapter{Đặt vấn đề}
Trong những năm gần đây, sự phát triển bùng nổ của các dịch vụ viễn thông, đặc biệt là các ứng dụng yêu cầu tốc độ cao, độ trễ thấp và khả năng quản lý linh hoạt, đã đặt ra những thách thức mới đối với hạ tầng mạng truyền thống. Kiến trúc mạng truy cập vô tuyến truyền thống (RAN) với các thiết bị và giao diện đóng độc quyền gây ra nhiều hạn chế về tính linh hoạt, khả năng tương tác và đổi mới công nghệ. Chính vì vậy, xu hướng chuyển dịch sang kiến trúc mạng truy cập vô tuyến mở (Open RAN) ngày càng thu hút sự quan tâm lớn từ cộng đồng nghiên cứu và các nhà cung cấp dịch vụ viễn thông hàng đầu thế giới.

Open RAN được kỳ vọng sẽ tái định hình ngành công nghiệp viễn thông nhờ vào việc mở các giao diện, phân tách chức năng mạng (như Central Unit - CU, Distributed Unit - DU, và Radio Unit - RU), đồng thời đưa trí tuệ nhân tạo (AI) vào sâu hơn trong việc quản lý và vận hành mạng lưới. Một thành phần cốt lõi giúp đạt được những lợi ích này chính là bộ điều khiển thông minh mạng vô tuyến (RAN Intelligent Controller - RIC) với hai phiên bản chính là Near-Real-Time RIC và Non-Real-Time RIC. Hai loại RIC này giúp quản lý và tối ưu các ứng dụng mạng (xApps và rApps) trong thời gian thực hoặc gần thực.

Mục tiêu của báo cáo này là làm rõ những ưu điểm của Open RAN so với mạng RAN truyền thống, đồng thời phân tích cụ thể vai trò của các thành phần quan trọng như RIC và các giao diện mở (E2, A1, O1). Qua đó, nhóm nghiên cứu mong muốn đề xuất một mô hình tối ưu hóa tài nguyên mạng hiệu quả bằng công nghệ học tăng cường sâu (DRL) nhằm đáp ứng các yêu cầu về hiệu năng, linh hoạt và khả năng mở rộng của mạng viễn thông thế hệ mới.

\chapter{Kiến thức cơ sở}
\section{Mobile Edge Computing (MEC) – Tính toán biên di động}


MEC là một kiến trúc tính toán phân tán, trong đó tài nguyên xử lý và lưu trữ đám mây được bố trí tại biên mạng di động, gần với phía người dùng hơn so với đám mây truyền thống. Ý tưởng chính của MEC là giảm khoảng cách vật lý giữa thiết bị người dùng (UE) và máy chủ thực thi tác vụ, qua đó giảm độ trễ truyền dẫn và tiết kiệm băng thông backhaul \cite{ris_latency}. Trong MEC, các UE có thể offload (tải lên) một phần hoặc toàn bộ tác vụ tính toán nặng (ví dụ: xử lý ảnh, video, phân tích dữ liệu IoT, AI...) đến máy chủ MEC đặt tại các trạm gốc hoặc điểm truy cập WiFi cục bộ. Máy chủ MEC, tuy có năng lực hạn chế hơn cloud trung tâm, nhưng nhờ vị trí gần UE nên có thể trả kết quả nhanh, đáp ứng yêu cầu độ trễ nghiêm ngặt của ứng dụng thời gian thực. 


Có hai chế độ offload phổ biến: (i) Offload toàn phần (binary offloading) – toàn bộ tác vụ được gửi lên MEC hoặc xử lý cục bộ, phù hợp với tác vụ nhỏ, không thể tách; (ii) Offload từng phần (partial offloading) – tác vụ được chia thành nhiều phần, một phần xử lý tại UE, phần còn lại gửi lên MEC, phù hợp với tác vụ lớn có thể song song hóa \cite{ris_latency}. MEC đã chứng minh hiệu quả trong giảm trễ so với điện toán đám mây truyền thống, nhưng vẫn gặp thách thức khi kênh vô tuyến không đảm bảo hoặc số lượng thiết bị lớn. Để nâng cao thông lượng offload, nhiều giải pháp bổ trợ đã được nghiên cứu: mạng di động không đồng nhất (HetNet) với các small-cell để giảm khoảng cách UE-AP; sử dụng massive MIMO tại trạm gốc để tăng phân tập và chống nhiễu; truyền ở băng tần mmWave hay THz để có băng thông rộng hơn cho offload; dùng UAV làm trạm di động để tạo đường truyền LOS linh hoạt. Tuy nhiên, các giải pháp này cũng có nhược điểm riêng (chi phí triển khai cao, kiến trúc phức tạp, tiêu thụ năng lượng lớn). Do đó, xuất hiện nhu cầu tìm kiếm giải pháp khác bổ trợ MEC hiệu quả, đặc biệt trong bối cảnh 6G.
\section{Chức năng của Near-Real-Time RIC và Non-Real-Time RIC}
Trong kiến trúc mạng truy cập vô tuyến mở (Open RAN), Bộ điều khiển RAN thông minh (RAN Intelligent Controller – RIC) được phân tách thành hai phần: RIC thời gian gần thực (Near-Real-Time RIC) và RIC thời gian không thực (Non-Real-Time RIC). Hai thành phần này phối hợp điều khiển và tối ưu mạng ở những quy mô thời gian khác nhau nhằm nâng cao hiệu năng của RAN. Cụ thể, Near-RT RIC đảm nhiệm việc điều khiển RAN với độ trễ thấp (từ khoảng 10 mili-giây đến <1 giây) \cite{Understanding_O-Ran}, còn Non-RT RIC phụ trách các tác vụ ở quy mô thời gian dài hơn (>1 giây, thường tính bằng giây, phút hoặc lâu hơn) \cite{Understanding_O-Ran}. Sự phân chia này cho phép tối ưu mạng ở cả thời gian thực ngắn hạn lẫn hoạch định dài hạn, tạo nên hệ thống điều khiển nhiều tầng cho RAN.

Near-Real-Time RIC (Near-RT RIC): Đây là thành phần RIC hoạt động gần thời gian thực, thường được triển khai trên hạ tầng điện toán biên hoặc cụm mạng khu vực gần với các nút RAN. Near-RT RIC có chức năng thu thập thông tin trạng thái mạng và thực thi các hành động điều khiển nhanh lên mạng vô tuyến với độ trễ yêu cầu dưới ~1 giây \cite{Understanding_O-Ran}. Theo đặc tả O-RAN, Near-RT RIC là một chức năng logic cho phép điều khiển và tối ưu tài nguyên RAN ở mức độ nhanh, thông qua việc thu thập dữ liệu chi tiết và tác động hành động lên các nút RAN qua giao diện E2 docs.o-ran-sc.org. Near-RT RIC thường xử lý các tác vụ như điều khiển truy cập vô tuyến và tài nguyên vô tuyến theo thời gian thực gần, ví dụ: điều phối lịch truyền dẫn, cân bằng tải giữa các cell, điều chỉnh tham số handover, điều khiển can nhiễu,… nhằm tối ưu hiệu suất thông lượng và chất lượng dịch vụ tức thời cho người dùng. Thành phần này tương tác trực tiếp với các nút mạng RAN (như O-DU, O-CU) qua giao diện E2 để nhận số liệu tình trạng (telemetry) và gửi chỉ thị điều khiển một cách liên tục. Đặc điểm quan trọng của Near-RT RIC là khả năng mở rộng chức năng qua các xApp – những ứng dụng plug-and-play chạy trên nền tảng Near-RT RIC để thực hiện các thuật toán điều khiển radio chuyên biệt \cite{Understanding_O-Ran}. Near-RT RIC cũng có cơ chế phối hợp và tránh xung đột giữa nhiều xApp khác nhau cùng tác động lên RAN (ví dụ: cơ chế quản lý message bus, lớp dữ liệu chia sẻ, và logic phân giải xung đột) để đảm bảo các quyết định điều khiển không mâu thuẫn \cite{Understanding_O-Ran}.

Non-Real-Time RIC (Non-RT RIC): Đây là thành phần RIC hoạt động ngoài thời gian thực chặt chẽ, nằm trong khối dịch vụ quản lý và điều hành (Service Management and Orchestration – SMO) ở trung tâm mạng hoặc đám mây. Non-RT RIC chịu trách nhiệm thực hiện các tác vụ quản lý, tối ưu RAN ở quy mô dài hạn hơn (trên 1 giây) \cite{Understanding_O-Ran}, bao gồm quản lý chính sách dịch vụ, phân tích hiệu năng, tối ưu cấu hình và hoạch định tài nguyên chiến lược cho mạng. Theo O-RAN Alliance, Non-RT RIC là một chức năng logic trong SMO hỗ trợ điều khiển/tối ưu RAN phi-thời-gian-thực, cung cấp khung AI/ML để huấn luyện và cập nhật mô hình, và truyền tải các hướng dẫn chính sách tới RIC gần thực \cite{Oran_overview} \cite{etsi_oranArchitecture}. Non-RT RIC được cấu thành bởi framework Non-RT RIC (nền tảng) và các ứng dụng rApp (các ứng dụng chạy trên Non-RT RIC). Nền tảng Non-RT RIC thực hiện việc kết thúc (terminate) giao diện A1 với Near-RT RIC, đồng thời phơi bày dịch vụ quản lý dữ liệu và ML cho các rApp thông qua giao diện nội bộ R1 \cite{etsi_oranArchitecture}. Các rApp (RAN applications) là những ứng dụng mô-đun chạy trên Non-RT RIC, sử dụng các dịch vụ mà nền tảng cung cấp để tạo ra các giá trị gia tăng cho vận hành RAN \cite{etsi_oranArchitecture}. Nhiệm vụ của rApp rất đa dạng, bao gồm: đề xuất và điều chỉnh chính sách điều khiển RAN, phân tích dữ liệu hiệu năng dài hạn, tối ưu cấu hình tham số, cũng như cung cấp thông tin bổ sung (enrichment information) cho các ứng dụng khác \cite{etsi_oranArchitecture}. Non-RT RIC gửi hướng dẫn chính sách và mục tiêu đến Near-RT RIC thông qua giao diện A1 (ví dụ: chính sách về phân bổ tài nguyên, mục tiêu QoS cần đạt, tham số ngưỡng sự kiện, v.v.), nhờ đó ảnh hưởng gián tiếp đến hành vi của các xApp trên Near-RT RIC \cite{etsi_oranArchitecture}. Ngược lại, Non-RT RIC cũng thu thập phản hồi từ mạng (thông qua dữ liệu O1 hoặc qua báo cáo từ Near-RT RIC) để đánh giá và điều chỉnh các chiến lược tối ưu. Có thể xem Non-RT RIC như “bộ não” ở tầng trên, vạch ra chiến lược dài hạn cho mạng, trong khi Near-RT RIC là “cánh tay tác động” ở tầng dưới thực thi các điều chỉnh nhanh theo chiến lược đó.
\section{Ứng dụng RIS trong MEC để giảm độ trễ}

Việc kết hợp RIS vào hệ thống MEC (tức RIS-aided MEC) được kỳ vọng nâng cao hiệu năng hệ thống trên nhiều phương diện. Thứ nhất, RIS giúp cải thiện chất lượng kênh truyền giữa UE và AP, từ đó tăng tốc độ offload dữ liệu từ UE lên edge server (ES)
\cite{ris_latency}
. Tốc độ offload cao cho phép UE gửi xong dữ liệu sớm hơn, ES xử lý sớm và trả kết quả sớm, nên giảm độ trễ tổng cho ứng dụng. Thứ hai, với sự hỗ trợ của RIS, MEC có thể phục vụ các UE ở vị trí trước đây sóng khó tới (góc khuất, vùng biên cell) mà vẫn đảm bảo băng thông, giúp mở rộng vùng phục vụ MEC mà không cần tăng công suất phát hay triển khai thêm trạm. Thứ ba, RIS có thể tiết kiệm năng lượng hệ thống: do kênh được cải thiện, UE có thể truyền ở công suất thấp hơn cho cùng lượng dữ liệu, hoặc ES không cần chờ nhiều lần truyền lại do lỗi, qua đó tổng năng lượng dùng để hoàn thành tác vụ giảm đi. 



Một ví dụ điển hình, công trình của P. Di Lorenzo và cs. (2022) đã nghiên cứu MEC động có RIS, trong đó UE liên tục sinh dữ liệu cần xử lý và kênh vô tuyến thay đổi theo thời gian. Kết quả cho thấy với RIS, có thể tối ưu hóa phối hợp cấu hình RIS, tham số truyền thông (phân bổ băng thông, công suất) và tài nguyên tính toán tại ES để đảm bảo độ trễ trung bình dưới ngưỡng trong khi tối thiểu hóa tiêu thụ năng lượng của toàn hệ (cả UE lẫn mạng)
\cite{mec}
. Tuy nhiên, tất cả các tối ưu kể trên giả định lý tưởng rằng thông tin kênh qua RIS và trạng thái hệ thống đều biết hoàn hảo và ngay lập tức. Như đã nêu, thực tế để đạt được điều đó cần các thủ tục điều khiển phức tạp. Phần tiếp theo, chúng tôi đi vào chi tiết mô hình hệ thống và các thủ tục này.
3. Mô hình hệ thống MEC h

% \chapter{Phương pháp tối ưu hoá ABR}
% \input{chapters/chapter03}

\chapter{Cài đặt phương pháp và thực nghiệm}
\section{Tổng quan về mô phỏng}

Nhằm minh họa rõ ràng và trực quan hóa các lợi ích từ việc ứng dụng kiến trúc Open RAN và điều khiển thông minh dựa trên học tăng cường sâu (DRL), nhóm đã quyết định lựa chọn công cụ mô phỏng mã nguồn mở RIMEDO-TS của dự án ONOS. RIMEDO-TS là bộ công cụ được phát triển nhằm mô phỏng và đánh giá hiệu suất các thuật toán quản lý tài nguyên và điều khiển thông minh trong môi trường Open RAN, đặc biệt nhấn mạnh vào khả năng tích hợp dễ dàng các xApp dựa trên AI và DRL.


\section{Giới thiệu về RIMEDO-TS}
RIMEDO-TS được phát triển bởi dự án ONOS (Open Network Operating System), cung cấp môi trường mô phỏng chi tiết và linh hoạt cho các giải pháp quản lý tài nguyên mạng vô tuyến trong các tình huống thực tế như cân bằng tải, tối ưu hóa tài nguyên vô tuyến và quản lý chuyển giao thông minh giữa các trạm phát gốc (gNB). Bộ mô phỏng này bao gồm các module chính như:
\begin{itemize}
    \item RAN Simulation Module: mô phỏng môi trường mạng vô tuyến với các nút mạng O-RAN như CU, DU, RU.
    \item Near-Real-Time RIC Module: mô phỏng bộ điều khiển thông minh Near-RT RIC cho phép chạy các ứng dụng quản lý mạng vô tuyến theo thời gian thực gần (xApp).
    \item xApp Framework: cung cấp các hàm API chuẩn hóa để phát triển, triển khai và đánh giá hiệu suất của các ứng dụng AI/DRL quản lý tài nguyên.
\end{itemize}
\section{Mô hình hàng đợi động và độ trễ offload}

Như đã mô tả, hệ thống MEC động có RIS được mô hình bằng hai loại hàng đợi: hàng đợi truyền tại UE và hàng đợi xử lý tại ES. Mục tiêu thường gặp là duy trì độ trễ trung bình hoặc xác suất trễ dưới một ngưỡng yêu cầu. Hãy xét một UE điển hình: tốc độ sinh công việc trung bình là $\lambda$ (bit/s), dung lượng kênh offload thay đổi theo cấu hình RIS và phân bổ tài nguyên mỗi slot. Nếu không tính overhead, công suất kênh UL của UE phụ thuộc vào SNR nhận tại AP, theo công thức Shannon: $R = B_k \log_2(1 + \text{SNR})$ bit/s/Hz (với SNR do kênh trực tiếp + phản xạ cộng lại)
\cite{ris_latency}
. Tuy nhiên, do có overhead, thời gian thực sự dành cho truyền data trong mỗi slot chỉ là $T-\tau$. Nếu $\tau$ quá lớn, dù kênh có dung lượng cao, UE cũng chỉ truyền được ít dữ liệu do thiếu thời gian. Ngược lại, nếu $\tau$ quá nhỏ (ít điều khiển), kênh có thể không tối ưu hoặc nhiều lỗi, làm giảm thông lượng thực tế. Bài toán điều khiển MEC/RIS do đó đòi hỏi tối ưu $\tau$ và các quyết định điều khiển để vừa đảm bảo tốc độ xử lý kịp đầu vào, vừa không lãng phí thời gian. 



Độ trễ mà UE cảm nhận bao gồm trễ xếp hàng tại UE, trễ truyền qua kênh, và trễ xử lý tại ES. Việc RIS tăng tốc kênh chủ yếu giảm phần trễ truyền. Nhưng nếu overhead điều khiển làm giảm thời gian truyền hoặc gây lỗi, độ trễ có thể tăng lại. Do đó, khi thiết kế giao thức, ta phải tích hợp overhead điều khiển vào tính toán trễ. Công thức (7) trong bài báo gốc cho độ trễ trung bình $D_k$ của UE $k$ liên hệ với hàng đợi tổng và tốc độ tới $\lambda_k$
\cite{ris_latency}
. Chi tiết phức tạp nhưng ý nghĩa: để $D_k$ nhỏ, phải có $\text{E}[X_k] \approx \lambda_k$ (tốc độ xử lý >= tốc độ sinh việc) và biến động hàng đợi thấp – tức system phải được điều khiển tối ưu và đủ tài nguyên.
\section{Kết quả và đánh giá thực nghiệm}

Sau khi triển khai và chạy mô phỏng, nhóm thu thập các chỉ số đánh giá hiệu năng của hệ thống mạng như:

\begin{itemize}
    \item Thông lượng mạng trung bình của người dùng.
    \item Tỷ lệ tải cân bằng giữa các gNB.
    \item Tỷ lệ chuyển giao (handover) thành công.
    \item Hiệu suất thuật toán DRL (độ hội tụ, reward trung bình mỗi episode).
\end{itemize}


Các kết quả mô phỏng ban đầu cho thấy rõ hiệu quả của việc sử dụng kiến trúc Open RAN kết hợp với các thuật toán DRL thông minh trong việc cải thiện hiệu năng hệ thống mạng vô tuyến, khẳng định tính khả thi và hiệu quả của phương pháp này trong việc quản lý tài nguyên và cân bằng tải trong mạng viễn thông tương lai.


\chapter{Kết luận}
Báo cáo đã trình bày chi tiết các vấn đề cốt lõi liên quan đến kiến trúc mạng truy cập vô tuyến mở (Open RAN), với việc tập trung vào vai trò của các thành phần quan trọng như RIC và các giao diện mở (E2, A1, O1). Thông qua nghiên cứu, rõ ràng thấy được các lợi ích nổi bật của Open RAN so với các kiến trúc mạng truyền thống, đặc biệt là khả năng linh hoạt, tương tác đa nhà cung cấp và triển khai các ứng dụng AI và DRL thông minh trong việc quản lý tài nguyên mạng.

Phần thực nghiệm đã được thực hiện bằng việc sử dụng bộ công cụ mô phỏng RIMEDO-TS, thể hiện rõ tính ứng dụng thực tiễn của kiến trúc Open RAN và công nghệ điều khiển thông minh DRL. Kết quả thực nghiệm chứng minh rằng việc triển khai các xApp thông minh dựa trên thuật toán DQN mang lại hiệu quả cao trong việc cân bằng tải, tối ưu hóa tài nguyên và nâng cao hiệu suất tổng thể của mạng.

Kết quả nghiên cứu và thực nghiệm khẳng định rằng kiến trúc Open RAN cùng với công nghệ trí tuệ nhân tạo và học tăng cường sâu có tiềm năng lớn trong việc tái định hình và nâng cấp mạng viễn thông hiện đại. Việc tiếp tục nghiên cứu sâu hơn và mở rộng các ứng dụng này trong tương lai sẽ giúp tối ưu hóa đáng kể hiệu năng và chi phí vận hành của các nhà cung cấp dịch vụ viễn thông, góp phần quan trọng vào sự phát triển của ngành công nghiệp viễn thông nói chung.



% \chapter*{Tài liệu tham khảo}
% \noindent [1] Shaka Player. https://shaka-player-demo.appspot.com/docs/api/index.html. Accessed: May 15, 2025.

\noindent [2] Adam, Giuseppe Samela. (2023). Adaptive Bitrate Streaming Evolved: WISH ABR and Bitmovin’s Player Integration. https://bitmovin.com/blog/wish-abr-adaptive-bitrate-streaming. Accessed: May 16, 2025.

\noindent [3] Vijaya Sagar Vinnakota. (2021). A TL;DR on ABR algorithms in MPEG DASH.
https://www.linkedin.com/pulse/tldr-abr-algorithms-mpeg-dash-vijaya-sagar-vinnakota. Accessed: May 17, 2025.

\noindent [4] Ziyu Zhong, Mufan Liu, Le Yang, Yifan Wang, Yiling Xu, Jenq-Neng Hwang. (2025). Video Streaming with Kairos: An MPC-Based ABR with Streaming-Aware Throughput Prediction. https://cs.paperswithcode.com/paper/video-streaming-with-kairos-an-mpc-based-abr. Accessed: May 19, 2025.

\noindent [5] Yueheng Li, Qianyuan Zheng, Zicheng Zhang, Hao Chen, Zhan Ma. (2025). Improving ABR Performance for Short Video Streaming Using
Multi-Agent Reinforcement Learning with Expert Guidance. https://arxiv.org/pdf/2304.04637. Accessed: May 22, 2025.

\noindent [6] Huang, Tianchi and Yao, Xin and Wu, Chenglei and Zhang, Rui-Xiao and Sun, Lifeng. (2018). Tiyuntsong: A Self-Play Reinforcement Learning Approach for ABR Video Streaming. https://github.com/thu-media/Tiyuntsong. Accessed: May 23, 2025.

\noindent [7] Lisa. What is WISH ABR? Revolutionizing Adaptive Streaming Quality.
https://www.coconut.co/articles/wish-abr-revolutionizing-adaptive-streaming-quality. Accessed: Jun 3, 2025.

\noindent [8] Version History of Shaka Player. https://github.com/shaka-project/shaka-player/releases?utm\_source=chatgpt.com. Accessed: Jun 4, 2025.



\printbibliography[heading=bibintoc]

\end{document}