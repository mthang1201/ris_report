\documentclass[12pt]{report}
\usepackage{polyglossia}
\setmainlanguage{vietnamese}
\usepackage{graphicx}
\graphicspath{{images/}}
\usepackage[a4paper]{geometry}
\usepackage{lipsum}
\usepackage{fancyhdr}
\usepackage{indentfirst}
% \setlength{\parindent}{15pt}
\pagestyle{fancy}
\fancyhead{}
\fancyhead[L]{\leftmark}
% \fancyfoot{}
% \fancyfoot[R]{\thepage}
% \fancyfoot[L]{Chương \thechapter}
% \fancyfoot[C]{Bùi Minh Thắng}
\fancyheadoffset[R]{0mm}
\usepackage{fontspec}
\setmainfont{Times New Roman}
\usepackage{tikz}
\usetikzlibrary{calc}
\usepackage{listings}
\usepackage{xcolor}
\usepackage{courier}


\usepackage{biblatex}
\addbibresource{refs.bib}

\title{
    {Thesis Title}\\
    {\large Institution Name}\\
    % {\includegraphics{UET_logo.png}}
}
\author{Bùi Minh Thắng}
\date{March 2025}

\begin{document}
% \fontsize{14pt}{18pt}\selectfont

\newgeometry{
    top=2.2cm,
    bottom=2.2cm,
    left=2.2cm,
    right=2.2cm
}

\begin{titlepage}
    \begin{tikzpicture}[overlay,remember picture]
    \draw [line width=0.5mm]
        ($ (current page.north west) + (2cm,-2cm) $)
        rectangle
        ($ (current page.south east) + (-2cm,2cm) $);
    \draw [line width=0.5m  m]
        ($ (current page.north west) + (1.9cm,-1.9cm) $)
        rectangle
        ($ (current page.south east) + (-1.9cm,1.9cm) $);
    \end{tikzpicture}
    \begin{center}
        \large\textbf{ĐẠI HỌC QUỐC GIA HÀ NỘI}\\
        \large\textbf{TRƯỜNG ĐẠI HỌC CÔNG NGHỆ}\\
        ----------------***---------------

        \vspace{3cm}
        {\large\textbf{BÁO CÁO THU HOẠCH HỌC PHẦN CÁC VẤN ĐỀ HIỆN ĐẠI CỦA TRUYỀN THÔNG VÀ MẠNG MÁY TÍNH}}

        \vspace{3cm}
        {\huge\textbf{Mô phỏng điều hướng lưu lượng trong O-RAN với xApp dùng Deep Reinforcement Learning}}

        \vfill
        \begin{table}[h]
            \centering
            \large
            \begin{tabular}{ll}
                \textbf{Họ và Tên:} & Bùi Minh Thắng - 23020646 \\
                \textbf{          } & Nguyễn Vũ Minh - 23020629 \\
                \textbf{          } & Ma Đức Minh - 23020626 \\
                \textbf{          } & Nguyễn Hoàng Tùng Dương - 21020182 \\
                \textbf{Người hướng dẫn:} & TS. Nguyễn Ngọc Tân \\
                \textbf{                } & CN. Nguyễn Thái Dương \\
            \end{tabular}
        \end{table}

        \vspace{4cm}
        \textbf{Hà Nội, 2025}
        
    \end{center}
\end{titlepage}

\newgeometry{
    top=2cm,
    bottom=2cm,
    left=3cm,
    right=2cm
}

\chapter*{Lời cam đoan}
Nhóm em xin cam đoan: Báo cáo nghiên cứu khoa học với đề tài “Mô phỏng điều hướng lưu lượng trong O-RAN với xApp dùng Deep Reinforcement Learning” này là của nhóm em. Những gì nhóm em viết ra không có sự sao chép từ các tài liệu, không sử dụng kết quả của người khác mà không trích dẫn cụ thể. Đây là công trình nghiên cứu tập thể nhóm em tự phát triển, không sao chép mã nguồn của người khác. Nếu vi phạm những điều trên, nhóm em xin chấp nhận tất cả những truy cứu về trách nhiệm theo quy định.

\vspace{2cm} % nếu cần tạo khoảng cách từ nội dung phía trên
\noindent
\hfill
\begin{minipage}{0.4\textwidth}
    \centering
    \textbf{Sinh viên}\\[1cm] % khoảng trống cho chữ ký
    Bùi Minh Thắng\\
    Nguyễn Vũ Minh\\
    Ma Đức Minh\\
    Nguyễn Hoàng Tùng Dương\\
\end{minipage}

\chapter*{Lời cảm ơn}
Lời đầu tiên, em xin được gửi lời cảm ơn chân thành tới Khoa Công nghệ Thông tin – Trường Đại học Công nghệ – Đại học Quốc gia Hà Nội đã tạo điều kiện thuận lợi để em được học tập, nghiên cứu và thực hiện đề tài này.

Em xin bày tỏ lòng biết ơn sâu sắc tới thầy Nguyễn Ngọc Tân và thầy Nguyễn Thái Dương đã tận tình hướng dẫn, hỗ trợ em trong suốt quá trình nghiên cứu và triển khai đề tài.

Bên cạnh đó, em xin được bày tỏ lòng biết ơn tới các thầy cô trong khoa đã tận tâm giảng dạy và trang bị cho em những kiến thức quý báu trong suốt quá trình học tập tại trường.

Cuối cùng, em xin chúc các thầy cô, các bạn luôn mạnh khỏe, hạnh phúc và gặt hái nhiều thành công trong cuộc sống.

\chapter*{Tóm tắt}
Báo cáo này trình bày quá trình phát triển và mô phỏng một xApp trong kiến trúc Open RAN, tập trung vào việc điều hướng lưu lượng mạng sử dụng Deep Reinforcement Learning (DRL). Chúng tôi đã xây dựng một xApp để tối ưu hóa việc phân phối lưu lượng mạng trong môi trường O-RAN, sử dụng mô hình DRL để cải thiện hiệu suất mạng. Bằng cách sử dụng mô phỏng ns-3, chúng tôi đã kiểm tra hiệu quả của xApp trong các tình huống thực tế, cho thấy khả năng cải thiện đáng kể trong việc quản lý lưu lượng và tối ưu hóa tài nguyên mạng. Báo cáo cũng cung cấp hướng dẫn chi tiết về cách phát triển xApp và tích hợp nó vào hệ thống O-RAN.

\textbf{Từ khoá:} Open RAN architecture, O-RAN RIC, xApp development, O-RAN use cases, ns-3 O-RAN simulation, DRL for traffic steering, RAN Intelligent Controller.

\tableofcontents

% % \listoffigures

% % \listoftables

% % \chapter*{Thuật ngữ}


\chapter{Đặt vấn đề}
Mạng không dây thế hệ sau 5G và hướng tới 6G dự kiến kích hoạt nhiều dịch vụ và ứng dụng mới (ví dụ: Công nghiệp 4.0, Internet vạn vật - IoT, xe tự hành), vốn đòi hỏi xử lý dữ liệu lớn với độ tin cậy cao và độ trễ đầu-cuối thấp. Để đáp ứng các yêu cầu này, tính toán biên di động (MEC) đã nổi lên như một công nghệ chủ chốt, cho phép đưa khả năng điện toán và lưu trữ đám mây đến gần biên mạng (ví dụ: tại các trạm truy cập không dây). Thay vì xử lý toàn bộ tác vụ trên thiết bị người dùng (UE) hoặc gửi lên đám mây trung tâm (với độ trễ cao), MEC cho phép offload các tác vụ nặng tới máy chủ biên gần UE để giảm độ trễ truyền và tiết kiệm năng lượng cho thiết bị. Nhiều nghiên cứu đã đề xuất các giải pháp tối ưu tài nguyên mạng cho MEC nhằm đảm bảo chất lượng dịch vụ (QoS) thỏa đáng cho người dùng, trong cả kịch bản tĩnh và động. 


Tuy nhiên, hiệu năng của hệ thống MEC phụ thuộc rất lớn vào chất lượng kết nối không dây giữa UE và điểm truy cập (AP) tích hợp máy chủ biên. Nếu kênh vô tuyến xấu (suy hao, chặn, nhiễu cao), tốc độ offload giảm mạnh, dẫn đến hàng đợi tác vụ ùn ứ và trễ xử lý tăng lên, làm mất lợi thế của MEC. Gần đây, công nghệ Reconfigurable Intelligent Surface (RIS) được nghiên cứu nhiều nhằm điều khiển môi trường truyền sóng không dây, cải thiện chất lượng kênh vật lý. RIS gồm một mảng bề mặt nhiều phần tử có thể điều chỉnh pha (và biên độ) phản xạ sóng điện từ, qua đó hướng tín hiệu đến nơi mong muốn và tăng cường chất lượng liên kết mà không cần truyền thêm công suất phát
. Việc tích hợp RIS trong hệ thống MEC hứa hẹn nâng cao tốc độ offload và mở rộng vùng phủ sóng hiệu quả của MEC, ngay cả khi kênh truyền thẳng UE-AP bị chặn hay suy hao nặng. Các nghiên cứu gần đây đã xem xét tối ưu hóa MEC có RIS (RIS-aided MEC) trong nhiều kịch bản: từ offload tĩnh đến động, từ tối ưu hóa truyền thống đến sử dụng học máy.



Tuy nhiên, một khoảng trống trong các nghiên cứu MEC/RIS trước đây là chưa xem xét đầy đủ các thủ tục điều khiển cần thiết để vận hành RIS trong hệ thống MEC cũng như tác động của chúng đến QoS người dùng cuối. Hầu hết các công trình giả định lý tưởng rằng RIS có thể điều chỉnh tức thì và hệ thống luôn có thông tin kênh hoàn hảo mà không tính đến chi phí thu thập thông tin đó. Trên thực tế, để sử dụng được RIS, hệ thống phải thực hiện một loạt thao tác điều khiển như ước lượng kênh (Channel Estimation - CE) cho các liên kết qua RIS, tối ưu phân bổ tài nguyên (Resource Allocation - RA) gồm lựa chọn cấu hình RIS, điều chỉnh công suất, băng thông truyền,..., và trao đổi tín hiệu điều khiển giữa các thực thể (AP, RIS, UE, máy chủ) để phối hợp vận hành. Những tác vụ điều khiển này tiêu tốn thời gian và tài nguyên của hệ thống – gọi chung là overhead điều khiển – và do đó ảnh hưởng trực tiếp đến độ trễ người dùng cảm nhận cũng như tiêu thụ năng lượng của mạng.

\chapter{Kiến thức cơ sở}
\section{So sánh kiến trúc RAN truyền thống và O-RAN}
Các mạng di động truyền thống trong lịch sử chủ yếu áp dụng kiến trúc RAN phân tán (Distributed RAN - D-RAN), trong đó mỗi trạm gốc tích hợp cả chức năng xử lý băng tần gốc (baseband) và chức năng vô tuyến. Mô hình này phù hợp với các thế hệ di động đầu tiên (2G/3G/4G), nhưng tồn tại nhiều hạn chế về khả năng mở rộng, tính tương tác giữa các nhà cung cấp và khả năng quản lý tập trung. Với sự ra đời của 5G và tương lai là 6G, ngành công nghiệp viễn thông đang chuyển dịch sang các giải pháp linh hoạt và định nghĩa bằng phần mềm. Open RAN (O-RAN) đưa ra một kiến trúc mở, mô-đun và tách rời bằng cách phân tách các chức năng RAN thành ba thành phần riêng biệt: CU, DU và RU, cùng với các giao diện tiêu chuẩn hóa \cite{oran-wg6-architecture}. Bài viết này phân tích các điểm khác biệt về kiến trúc và những hàm ý giữa D-RAN và O-RAN.

Trong D-RAN, mỗi trạm thu phát (cell site) chứa đồng thời một Bộ xử lý băng tần gốc (BBU) và một Bộ vô tuyến từ xa (RRU). BBU thực hiện toàn bộ xử lý tầng baseband, bao gồm PHY, MAC, RLC và các tầng giao thức cao hơn, trong khi RRU đảm nhận các chức năng đầu vào/ra RF tương tự. Sự tích hợp này giúp đơn giản hóa việc đồng bộ hóa và giảm độ trễ đường truyền fronthaul, nhưng lại dẫn đến hệ thống cứng nhắc và phụ thuộc vào nhà cung cấp. Do phần cứng và phần mềm gắn kết chặt chẽ, việc mở rộng hoặc nâng cấp hệ thống thường đòi hỏi phải thay thế toàn bộ các thành phần độc quyền. Ngoài ra, khả năng tối ưu hóa mạng và chia sẻ tài nguyên giữa các trạm bị hạn chế do xử lý vẫn diễn ra độc lập tại từng vị trí \cite{3gpp-tr38.801}.

O-RAN khác biệt với cách tiếp cận nguyên khối của D-RAN bằng việc phân tách chức năng rõ ràng giữa RU, DU và CU. RU đảm nhận xử lý PHY thấp và RF, DU chịu trách nhiệm xử lý PHY cao, MAC và RLC, trong khi CU điều khiển các tầng SDAP, PDCP và RRC. Các thành phần này giao tiếp thông qua các giao diện mở: giao diện F1 giữa CU và DU, và giao diện fronthaul mở (thường là chia tách 7.2x) giữa DU và RU \cite{khurshid2024oran}. Tính mô-đun này cho phép các thiết bị từ nhiều nhà cung cấp khác nhau tương tác, đồng thời hỗ trợ ảo hóa các chức năng CU/DU trên phần cứng thương mại thông dụng \cite{ericsson-openran}.

Một điểm khác biệt lớn giữa D-RAN và O-RAN nằm ở tính linh hoạt khi triển khai. Trong khi D-RAN đồng vị tất cả xử lý tại trạm thu phát, thì O-RAN hỗ trợ việc tập trung hóa CU tại các trung tâm dữ liệu khu vực và phân phối DU gần biên mạng. Sự phân tách này cho phép phân bổ tài nguyên động, giảm chi phí đầu tư (CAPEX) và nâng cao khả năng mở rộng. Hơn nữa, O-RAN còn hỗ trợ các chức năng tiên tiến như bộ điều khiển RAN thông minh (RIC), cho phép tối ưu hóa dựa trên AI/ML gần như theo thời gian thực \cite{oran-wg6-architecture}.

Tuy có nhiều ưu điểm, O-RAN cũng đặt ra thách thức trong việc tích hợp do môi trường đa nhà cung cấp và yêu cầu nghiêm ngặt về đường truyền fronthaul. Các liên kết có tốc độ cao và độ trễ thấp là điều kiện bắt buộc để đảm bảo hiệu năng trong kiến trúc chia tách. Ngược lại, D-RAN đơn giản hơn trong triển khai và đã được kiểm chứng về độ tin cậy, nhưng thiếu sự linh hoạt và tính mở cần thiết cho các trường hợp sử dụng mới như chia sẻ mạng (network slicing) và mạng 5G riêng (private 5G) \cite{ericsson-openran}.

Kiến trúc D-RAN truyền thống đã đóng vai trò là nền tảng của mạng di động trong nhiều thập kỷ, mang lại sự đơn giản và độ ổn định cho các thế hệ mạng di động trước. Tuy nhiên, nhu cầu ngày càng tăng về tính linh hoạt, hiệu quả chi phí và phân biệt dịch vụ đang thúc đẩy quá trình chuyển đổi sang hệ thống mở và tách rời. O-RAN, với việc phân tách chức năng và các giao diện tiêu chuẩn, đại diện cho một sự chuyển đổi mô hình hướng tới giải pháp RAN gốc đám mây và trung lập với nhà cung cấp. Mặc dù vẫn còn tồn tại các thách thức về tích hợp và tối ưu hóa hiệu năng, nhưng những lợi ích tiềm năng của O-RAN đang định vị nó như một yếu tố then chốt trong việc phát triển mạng di động tương lai \cite{khurshid2024oran}.
\section{Chức năng của Near-Real-Time RIC và Non-Real-Time RIC}
Trong kiến trúc mạng truy cập vô tuyến mở (Open RAN), Bộ điều khiển RAN thông minh (RAN Intelligent Controller – RIC) được phân tách thành hai phần: RIC thời gian gần thực (Near-Real-Time RIC) và RIC thời gian không thực (Non-Real-Time RIC). Hai thành phần này phối hợp điều khiển và tối ưu mạng ở những quy mô thời gian khác nhau nhằm nâng cao hiệu năng của RAN. Cụ thể, Near-RT RIC đảm nhiệm việc điều khiển RAN với độ trễ thấp (từ khoảng 10 mili-giây đến <1 giây) \cite{Understanding_O-Ran}, còn Non-RT RIC phụ trách các tác vụ ở quy mô thời gian dài hơn (>1 giây, thường tính bằng giây, phút hoặc lâu hơn) \cite{Understanding_O-Ran}. Sự phân chia này cho phép tối ưu mạng ở cả thời gian thực ngắn hạn lẫn hoạch định dài hạn, tạo nên hệ thống điều khiển nhiều tầng cho RAN.

Near-Real-Time RIC (Near-RT RIC): Đây là thành phần RIC hoạt động gần thời gian thực, thường được triển khai trên hạ tầng điện toán biên hoặc cụm mạng khu vực gần với các nút RAN. Near-RT RIC có chức năng thu thập thông tin trạng thái mạng và thực thi các hành động điều khiển nhanh lên mạng vô tuyến với độ trễ yêu cầu dưới ~1 giây \cite{Understanding_O-Ran}. Theo đặc tả O-RAN, Near-RT RIC là một chức năng logic cho phép điều khiển và tối ưu tài nguyên RAN ở mức độ nhanh, thông qua việc thu thập dữ liệu chi tiết và tác động hành động lên các nút RAN qua giao diện E2 docs.o-ran-sc.org. Near-RT RIC thường xử lý các tác vụ như điều khiển truy cập vô tuyến và tài nguyên vô tuyến theo thời gian thực gần, ví dụ: điều phối lịch truyền dẫn, cân bằng tải giữa các cell, điều chỉnh tham số handover, điều khiển can nhiễu,… nhằm tối ưu hiệu suất thông lượng và chất lượng dịch vụ tức thời cho người dùng. Thành phần này tương tác trực tiếp với các nút mạng RAN (như O-DU, O-CU) qua giao diện E2 để nhận số liệu tình trạng (telemetry) và gửi chỉ thị điều khiển một cách liên tục. Đặc điểm quan trọng của Near-RT RIC là khả năng mở rộng chức năng qua các xApp – những ứng dụng plug-and-play chạy trên nền tảng Near-RT RIC để thực hiện các thuật toán điều khiển radio chuyên biệt \cite{Understanding_O-Ran}. Near-RT RIC cũng có cơ chế phối hợp và tránh xung đột giữa nhiều xApp khác nhau cùng tác động lên RAN (ví dụ: cơ chế quản lý message bus, lớp dữ liệu chia sẻ, và logic phân giải xung đột) để đảm bảo các quyết định điều khiển không mâu thuẫn \cite{Understanding_O-Ran}.

Non-Real-Time RIC (Non-RT RIC): Đây là thành phần RIC hoạt động ngoài thời gian thực chặt chẽ, nằm trong khối dịch vụ quản lý và điều hành (Service Management and Orchestration – SMO) ở trung tâm mạng hoặc đám mây. Non-RT RIC chịu trách nhiệm thực hiện các tác vụ quản lý, tối ưu RAN ở quy mô dài hạn hơn (trên 1 giây) \cite{Understanding_O-Ran}, bao gồm quản lý chính sách dịch vụ, phân tích hiệu năng, tối ưu cấu hình và hoạch định tài nguyên chiến lược cho mạng. Theo O-RAN Alliance, Non-RT RIC là một chức năng logic trong SMO hỗ trợ điều khiển/tối ưu RAN phi-thời-gian-thực, cung cấp khung AI/ML để huấn luyện và cập nhật mô hình, và truyền tải các hướng dẫn chính sách tới RIC gần thực \cite{Oran_overview} \cite{etsi_oranArchitecture}. Non-RT RIC được cấu thành bởi framework Non-RT RIC (nền tảng) và các ứng dụng rApp (các ứng dụng chạy trên Non-RT RIC). Nền tảng Non-RT RIC thực hiện việc kết thúc (terminate) giao diện A1 với Near-RT RIC, đồng thời phơi bày dịch vụ quản lý dữ liệu và ML cho các rApp thông qua giao diện nội bộ R1 \cite{etsi_oranArchitecture}. Các rApp (RAN applications) là những ứng dụng mô-đun chạy trên Non-RT RIC, sử dụng các dịch vụ mà nền tảng cung cấp để tạo ra các giá trị gia tăng cho vận hành RAN \cite{etsi_oranArchitecture}. Nhiệm vụ của rApp rất đa dạng, bao gồm: đề xuất và điều chỉnh chính sách điều khiển RAN, phân tích dữ liệu hiệu năng dài hạn, tối ưu cấu hình tham số, cũng như cung cấp thông tin bổ sung (enrichment information) cho các ứng dụng khác \cite{etsi_oranArchitecture}. Non-RT RIC gửi hướng dẫn chính sách và mục tiêu đến Near-RT RIC thông qua giao diện A1 (ví dụ: chính sách về phân bổ tài nguyên, mục tiêu QoS cần đạt, tham số ngưỡng sự kiện, v.v.), nhờ đó ảnh hưởng gián tiếp đến hành vi của các xApp trên Near-RT RIC \cite{etsi_oranArchitecture}. Ngược lại, Non-RT RIC cũng thu thập phản hồi từ mạng (thông qua dữ liệu O1 hoặc qua báo cáo từ Near-RT RIC) để đánh giá và điều chỉnh các chiến lược tối ưu. Có thể xem Non-RT RIC như “bộ não” ở tầng trên, vạch ra chiến lược dài hạn cho mạng, trong khi Near-RT RIC là “cánh tay tác động” ở tầng dưới thực thi các điều chỉnh nhanh theo chiến lược đó. 
\section{Ứng dụng RIS trong MEC để giảm độ trễ}

Việc kết hợp RIS vào hệ thống MEC (tức RIS-aided MEC) được kỳ vọng nâng cao hiệu năng hệ thống trên nhiều phương diện. Thứ nhất, RIS giúp cải thiện chất lượng kênh truyền giữa UE và AP, từ đó tăng tốc độ offload dữ liệu từ UE lên edge server (ES)
\cite{ris_latency}
. Tốc độ offload cao cho phép UE gửi xong dữ liệu sớm hơn, ES xử lý sớm và trả kết quả sớm, nên giảm độ trễ tổng cho ứng dụng. Thứ hai, với sự hỗ trợ của RIS, MEC có thể phục vụ các UE ở vị trí trước đây sóng khó tới (góc khuất, vùng biên cell) mà vẫn đảm bảo băng thông, giúp mở rộng vùng phục vụ MEC mà không cần tăng công suất phát hay triển khai thêm trạm. Thứ ba, RIS có thể tiết kiệm năng lượng hệ thống: do kênh được cải thiện, UE có thể truyền ở công suất thấp hơn cho cùng lượng dữ liệu, hoặc ES không cần chờ nhiều lần truyền lại do lỗi, qua đó tổng năng lượng dùng để hoàn thành tác vụ giảm đi. 



Một ví dụ điển hình, công trình của P. Di Lorenzo và cs. (2022) đã nghiên cứu MEC động có RIS, trong đó UE liên tục sinh dữ liệu cần xử lý và kênh vô tuyến thay đổi theo thời gian. Kết quả cho thấy với RIS, có thể tối ưu hóa phối hợp cấu hình RIS, tham số truyền thông (phân bổ băng thông, công suất) và tài nguyên tính toán tại ES để đảm bảo độ trễ trung bình dưới ngưỡng trong khi tối thiểu hóa tiêu thụ năng lượng của toàn hệ (cả UE lẫn mạng)
\cite{mec}
. Tuy nhiên, tất cả các tối ưu kể trên giả định lý tưởng rằng thông tin kênh qua RIS và trạng thái hệ thống đều biết hoàn hảo và ngay lập tức. Như đã nêu, thực tế để đạt được điều đó cần các thủ tục điều khiển phức tạp. Phần tiếp theo, chúng tôi đi vào chi tiết mô hình hệ thống và các thủ tục này.
3. Mô hình hệ thống MEC h

% \chapter{Phương pháp tối ưu hoá ABR}
% \input{chapters/chapter03}

\chapter{Cài đặt phương pháp và thực nghiệm}
\section{Cấu trúc hệ thống và các thành phần}


Hệ thống xem xét gồm các thiết bị người dùng (UE) muốn offload tác vụ lên một máy chủ biên (Edge Server - ES) đặt tại trạm Access Point (AP) gần đó; AP này được hỗ trợ bởi một RIS nhằm cải thiện kênh liên lạc (Hình 1). Giả sử AP có trang bị $M$ ăng-ten thu phát (một trạm nhiều ăng-ten hoặc trạm MIMO), mỗi UE có một ăng-ten, và RIS có $N$ phần tử phản xạ có thể điều chỉnh (với $N$ lớn) \cite{ris_latency}
. Tập các UE được ký hiệu $ \mathcal{U}$, tập các ăng-ten AP ký hiệu $ \mathcal{M}$
. Các UE có thể chia sẻ phổ bằng kỹ thuật ghép kênh phân chia tần số (FDM), nghĩa là mỗi UE được cấp một băng tần con riêng để tránh giao thoa với nhau
. Tổng băng thông hệ thống dành cho đường truyền MEC là $B$ Hz, phân chia cho $K$ UE: mỗi UE $k$ dùng băng thông $B_k$ Hz với $\sum_{k=1}^K B_k = B$ \cite{ris_latency}. 


Hoạt động của MEC diễn ra theo chuỗi thời gian rời rạc: thời gian được chia thành các slot bằng nhau độ dài $T$ (giây) đánh số $t = 1, 2, ...$
. Mỗi slot $T$ được chia thành hai phần: phần điều khiển (overhead điều khiển) và phần payload (dùng để truyền dữ liệu offload)
. Ký hiệu $\tau$ (tau) là tổng thời lượng phần điều khiển trong mỗi slot, khi đó thời gian còn lại cho truyền dữ liệu là $T - \tau$
\cite{ris_latency}
. Thông thường, $T$ được chọn tương ứng với thời gian hiệu dụng kênh (coherence time) sao cho kênh có thể coi là không đổi trong suốt một slot
. Như vậy, nếu kênh thay đổi chậm, ta có thể dùng slot dài để giảm tần suất điều khiển; nếu kênh thay đổi nhanh (ví dụ UE di chuyển tốc độ cao), slot phải ngắn để kịp theo kênh, nhưng chi phí là overhead sẽ chiếm tỷ lệ lớn hơn. 

\begin{figure}[H]
  \centering
  \includegraphics[width=0.8\textwidth]{images/f1.png}
  \caption{Mô hình hệ thống RIS-aided MEC}
    \label{fig:my-image}
\end{figure}



 Máy chủ MEC đặt tại AP chịu trách nhiệm xử lý các tác vụ nhận từ UE. Mỗi UE có một hàng đợi cục bộ để lưu trữ dữ liệu công việc chưa truyền, trong khi ES có một hàng đợi từ xa để lưu các dữ liệu đã nhận chờ xử lý
. Mô hình hàng đợi động: giả sử tại mỗi slot, UE $k$ sinh ra $A_k(t)$ (bit) dữ liệu mới vào hàng đợi cục bộ. Nếu UE truyền được $X_k(t)$ (bit) trong slot đó lên AP thành công, hàng đợi cục bộ sẽ giảm bớt tương ứng. Hàng đợi cục bộ của UE $k$ cập nhật theo công thức: $Q_k^\text{local}(t+1) = \max{Q_k^\text{local}(t) - X_k(t), 0} + A_k(t)$
\cite{ris_latency}
. Tại phía ES, nếu ES xử lý $Y_k(t)$ (bit) cho UE $k$ trong slot (sử dụng tài nguyên CPU), hàng đợi từ xa giảm đi, và đồng thời hàng đợi từ xa tăng thêm chính lượng $X_k(t)$ đã nhận vào. Biểu diễn: $Q_k^\text{ES}(t+1) = \max{Q_k^\text{ES}(t) - Y_k(t), 0} + X_k(t)$
. Độ trễ offload trung bình mà UE $k$ cảm nhận có thể được tính từ hàng đợi tổng (gộp local + ES) theo lý thuyết hàng đợi
, chẳng hạn khi hệ ổn định và $A_k(t)$ ổn định quanh tốc độ trung bình $\lambda_k$ (bit/s), độ trễ trung bình $D_k = \frac{Q_k^\text{local} + Q_k^\text{ES}}{\lambda_k}$
\cite{ris_latency}
.




\section{Giới thiệu về RIMEDO-TS}
RIMEDO-TS được phát triển bởi dự án ONOS (Open Network Operating System), cung cấp môi trường mô phỏng chi tiết và linh hoạt cho các giải pháp quản lý tài nguyên mạng vô tuyến trong các tình huống thực tế như cân bằng tải, tối ưu hóa tài nguyên vô tuyến và quản lý chuyển giao thông minh giữa các trạm phát gốc (gNB). Bộ mô phỏng này bao gồm các module chính như:
\begin{itemize}
    \item RAN Simulation Module: mô phỏng môi trường mạng vô tuyến với các nút mạng O-RAN như CU, DU, RU.
    \item Near-Real-Time RIC Module: mô phỏng bộ điều khiển thông minh Near-RT RIC cho phép chạy các ứng dụng quản lý mạng vô tuyến theo thời gian thực gần (xApp).
    \item xApp Framework: cung cấp các hàm API chuẩn hóa để phát triển, triển khai và đánh giá hiệu suất của các ứng dụng AI/DRL quản lý tài nguyên.
\end{itemize}
\section{Triển khai ứng dụng DRL (xApp)}

Nhóm tiến hành triển khai một ứng dụng DRL (Deep Reinforcement Learning) mẫu dưới dạng xApp, cụ thể sử dụng thuật toán DQN (Deep Q-Network) để tối ưu hóa cân bằng tải giữa các trạm phát gốc. Các bước gồm:

\begin{itemize}
    \item Xây dựng hoặc sử dụng lại module DQN agent mẫu có sẵn trong thư mục mẫu của RIMEDO-TS.
    \item Đăng ký và kích hoạt xApp trên nền tảng Near-RT RIC trong mô phỏng.
    \item Cấu hình các tham số huấn luyện và kiểm thử của thuật toán DQN, bao gồm số lượng episodes, learning rate, discount factor, và exploration-exploitation parameters.
\end{itemize}
\section{Kết quả và đánh giá thực nghiệm}

Sau khi triển khai và chạy mô phỏng, nhóm thu thập các chỉ số đánh giá hiệu năng của hệ thống mạng như:

\begin{itemize}
    \item Thông lượng mạng trung bình của người dùng.
    \item Tỷ lệ tải cân bằng giữa các gNB.
    \item Tỷ lệ chuyển giao (handover) thành công.
    \item Hiệu suất thuật toán DRL (độ hội tụ, reward trung bình mỗi episode).
\end{itemize}


Các kết quả mô phỏng ban đầu cho thấy rõ hiệu quả của việc sử dụng kiến trúc Open RAN kết hợp với các thuật toán DRL thông minh trong việc cải thiện hiệu năng hệ thống mạng vô tuyến, khẳng định tính khả thi và hiệu quả của phương pháp này trong việc quản lý tài nguyên và cân bằng tải trong mạng viễn thông tương lai.


\chapter{Kết luận}
\input{chapters/chapter05}

% \chapter*{Tài liệu tham khảo}
% \noindent [1] Shaka Player. https://shaka-player-demo.appspot.com/docs/api/index.html. Accessed: May 15, 2025.

\noindent [2] Adam, Giuseppe Samela. (2023). Adaptive Bitrate Streaming Evolved: WISH ABR and Bitmovin’s Player Integration. https://bitmovin.com/blog/wish-abr-adaptive-bitrate-streaming. Accessed: May 16, 2025.

\noindent [3] Vijaya Sagar Vinnakota. (2021). A TL;DR on ABR algorithms in MPEG DASH.
https://www.linkedin.com/pulse/tldr-abr-algorithms-mpeg-dash-vijaya-sagar-vinnakota. Accessed: May 17, 2025.

\noindent [4] Ziyu Zhong, Mufan Liu, Le Yang, Yifan Wang, Yiling Xu, Jenq-Neng Hwang. (2025). Video Streaming with Kairos: An MPC-Based ABR with Streaming-Aware Throughput Prediction. https://cs.paperswithcode.com/paper/video-streaming-with-kairos-an-mpc-based-abr. Accessed: May 19, 2025.

\noindent [5] Yueheng Li, Qianyuan Zheng, Zicheng Zhang, Hao Chen, Zhan Ma. (2025). Improving ABR Performance for Short Video Streaming Using
Multi-Agent Reinforcement Learning with Expert Guidance. https://arxiv.org/pdf/2304.04637. Accessed: May 22, 2025.

\noindent [6] Huang, Tianchi and Yao, Xin and Wu, Chenglei and Zhang, Rui-Xiao and Sun, Lifeng. (2018). Tiyuntsong: A Self-Play Reinforcement Learning Approach for ABR Video Streaming. https://github.com/thu-media/Tiyuntsong. Accessed: May 23, 2025.

\noindent [7] Lisa. What is WISH ABR? Revolutionizing Adaptive Streaming Quality.
https://www.coconut.co/articles/wish-abr-revolutionizing-adaptive-streaming-quality. Accessed: Jun 3, 2025.

\noindent [8] Version History of Shaka Player. https://github.com/shaka-project/shaka-player/releases?utm\_source=chatgpt.com. Accessed: Jun 4, 2025.



\printbibliography[heading=bibintoc]

\end{document}