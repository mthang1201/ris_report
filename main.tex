\documentclass[12pt]{report}
\usepackage{polyglossia}
\setmainlanguage{vietnamese}
\usepackage{graphicx}
\graphicspath{{images/}}
\usepackage[a4paper]{geometry}
\usepackage{lipsum}
\usepackage{fancyhdr}
\usepackage{indentfirst}
\pagestyle{fancy}
\fancyhead{}
\fancyhead[L]{\leftmark}
% \fancyfoot{}
% \fancyfoot[R]{\thepage}
% \fancyfoot[L]{Chương \thechapter}
% \fancyfoot[C]{Bùi Minh Thắng}
\fancyheadoffset[R]{0mm}
\usepackage{fontspec}
\setmainfont{Times New Roman}
\usepackage{tikz}
\usetikzlibrary{calc}
\usepackage{listings}
\usepackage{xcolor}
\usepackage{courier}


% \usepackage{biblatex}
% \addbibresource{references.bib}

\title{
    {Thesis Title}\\
    {\large Institution Name}\\
    % {\includegraphics{UET_logo.png}}
}
\author{Bùi Minh Thắng}
\date{March 2025}

\begin{document}
% \fontsize{14pt}{18pt}\selectfont

\newgeometry{
    top=2.2cm,
    bottom=2.2cm,
    left=2.2cm,
    right=2.2cm
}

\begin{titlepage}
    \begin{tikzpicture}[overlay,remember picture]
    \draw [line width=0.5mm]
        ($ (current page.north west) + (2cm,-2cm) $)
        rectangle
        ($ (current page.south east) + (-2cm,2cm) $);
    \draw [line width=0.5m  m]
        ($ (current page.north west) + (1.9cm,-1.9cm) $)
        rectangle
        ($ (current page.south east) + (-1.9cm,1.9cm) $);
    \end{tikzpicture}
    \begin{center}
        \large\textbf{ĐẠI HỌC QUỐC GIA HÀ NỘI}\\
        \large\textbf{TRƯỜNG ĐẠI HỌC CÔNG NGHỆ}\\
        ----------------***---------------

        \vspace{3cm}
        {\large\textbf{BÁO CÁO THU HOẠCH HỌC PHẦN CÁC VẤN ĐỀ HIỆN ĐẠI CỦA TRUYỀN THÔNG VÀ MẠNG MÁY TÍNH}}

        \vspace{3cm}
        {\huge\textbf{Các Khía Cạnh Điều Khiển trong Hệ Thống MEC Hỗ Trợ bởi RIS cho Ứng Dụng Độ Trễ Nghiêm Ngặt}}

        \vfill
        \begin{table}[h]
            \centering
            \large
            \begin{tabular}{ll}
                \textbf{Họ và Tên:} & Bùi Minh Thắng - 23020646 \\
                \textbf{          } & Nguyễn Vũ Minh - 23020629 \\
                \textbf{          } & Ma Đức Minh - 23020626 \\
                \textbf{          } & Nguyễn Hoàng Tùng Dương - 21020182 \\
                \textbf{Người hướng dẫn:} & TS. Nguyễn Ngọc Tân \\
                \textbf{                } & CN. Nguyễn Thái Dương \\
            \end{tabular}
        \end{table}

        \vspace{4cm}
        \textbf{Hà Nội, 2025}
        
    \end{center}
\end{titlepage}

\newgeometry{
    top=2cm,
    bottom=2cm,
    left=3cm,
    right=2cm
}

\chapter*{Lời cam đoan}
Em xin cam đoan: Báo cáo thu hoạch thực tập Viettel Digital Talent 2025 với đề tài “Áp dụng Machine Learning để tối ưu thuật toán ABR cho Shaka Player trên nền tảng Web” này là của em. Những gì em viết ra không có sự sao chép từ các tài liệu, không sử dụng kết quả của người khác mà không trích dẫn cụ thể. Đây là công trình nghiên cứu cá nhân em tự phát triển, không sao chép mã nguồn của người khác. Nếu vi phạm những điều trên, em xin chấp nhận tất cả những truy cứu về trách nhiệm theo quy định.

\vspace{2cm} % nếu cần tạo khoảng cách từ nội dung phía trên
\noindent
\hfill
\begin{minipage}{0.4\textwidth}
    \centering
    \textbf{Thực tập sinh}\\[1cm] % khoảng trống cho chữ ký
    Bùi Minh Thắng
\end{minipage}

\chapter*{Lời cảm ơn}
Em xin bày tỏ lòng biết ơn chân thành tới anh Lê Hoàng và anh Nguyễn Đức Vượng đã tận tình hướng dẫn, hỗ trợ em trong suốt quá trình nghiên cứu và thực hiện đề tài này.

Em cũng xin gửi lời cảm ơn sâu sắc tới các anh KOLs và mentor từ Viettel đã nhiệt tình giảng dạy, chia sẻ kiến thức và kinh nghiệm thực tiễn về Software Engineering trong suốt hai tháng vừa qua, giúp em có thêm nền tảng chuyên môn vững chắc để hoàn thành đề tài một cách tốt nhất.

Cuối cùng, em xin chúc các anh, các chị luôn dồi dào sức khỏe, thành công và hạnh phúc trong cuộc sống.

% Lời đầu tiên, em xin được gửi lời cảm ơn chân thành tới Khoa Công nghệ Thông tin – Trường Đại học Công nghệ – Đại học Quốc gia Hà Nội đã tạo điều kiện thuận lợi để em được học tập, nghiên cứu và thực hiện đề tài này.

% Em xin bày tỏ lòng biết ơn sâu sắc tới anh Lê Hoàng và anh Nguyễn Đức Vượng đã tận tình hướng dẫn, hỗ trợ em trong suốt quá trình nghiên cứu và triển khai đề tài.

% Em cũng xin cảm ơn các anh KOLs và mentor từ Viettel đã nhiệt tình giảng dạy, chia sẻ kiến thức và kinh nghiệm thực tiễn về Software Engineering trong suốt hai tháng vừa qua, giúp em có thêm nền tảng chuyên môn vững chắc để hoàn thành đề tài một cách tốt nhất.

% Bên cạnh đó, em xin được bày tỏ lòng biết ơn tới các thầy cô trong khoa đã tận tâm giảng dạy và trang bị cho em những kiến thức quý báu trong suốt quá trình học tập tại trường.

% Cuối cùng, em xin chúc các thầy cô, anh chị luôn mạnh khỏe, hạnh phúc và gặt hái nhiều thành công trong cuộc sống.

\chapter*{Tóm tắt}
Trong bối cảnh chất lượng trải nghiệm người dùng (QoE) ngày càng đóng vai trò then chốt đối với các nền tảng phát video trực tuyến, các thuật toán Adaptive Bitrate (ABR) mặc định trong các trình phát như Shaka Player thường bộc lộ nhiều hạn chế khi phải thích ứng với điều kiện mạng thay đổi liên tục. Các thuật toán này chủ yếu dựa trên các quy tắc cố định, thiếu khả năng cân bằng linh hoạt giữa nhiều yếu tố như độ mượt, độ trễ và mức sử dụng băng thông. Do đó, việc ứng dụng các phương pháp tối ưu hóa hiện đại như học máy hoặc mô hình chi phí trọng số trở nên cần thiết. Trong báo cáo này, dự án trình bày tổng quan về Shaka Player — một thư viện JavaScript mã nguồn mở do Google phát triển, hỗ trợ DASH và HLS — cùng với nguyên lý hoạt động của cơ chế ABR. Sau đó sẽ điểm qua một số thuật toán ABR nổi bật hiện nay như BOLA, MPC và Reinforcement Learning-based ABR, đồng thời phân tích lý do lựa chọn thuật toán WISH (Weighted-Sum Model for HTTP Adaptive Streaming) để giải quyết bài toán tối ưu chất lượng truyền phát. WISH là một mô hình tối ưu dựa trên hàm tổng trọng số, trong đó nhiều yếu tố ảnh hưởng đến QoE như bitrate, độ mượt, số lần gián đoạn và độ trễ được kết hợp với các trọng số linh hoạt nhằm đưa ra quyết định lựa chọn bitrate tại từng thời điểm sao cho tổng thể trải nghiệm người dùng được tối ưu hóa. Dự án hiện thực hóa thuật toán WISH trong một ứng dụng phát video trực tuyến sử dụng React.js cho frontend và Shaka Player cho trình phát video phía client. Trong quá trình triển khai, dự án cố gắng đưa mô hình WISH tích hợp vào hệ thống logic ABR của Shaka Player nhằm thay thế thuật toán mặc định nhưng chưa đạt được thành tựu. Báo cáo kết luận rằng việc áp dụng mô hình WISH trong kiến trúc dựa trên Shaka Player là một hướng tiếp cận khả thi, nhưng cần được nghiên cứu thêm.

\textbf{Từ khoá:} Machine Learning, ABR, QoE, Shaka Player, WISH, Reinforcement Learning-based ABR, web, mô hình tổng trọng số, phát video trực tuyến, DASH.

\tableofcontents

% % \listoffigures

% % \listoftables

% % \chapter*{Thuật ngữ}


\chapter{Đặt vấn đề}
% Trong những năm gần đây, sự phát triển bùng nổ của các dịch vụ viễn thông, đặc biệt là các ứng dụng yêu cầu tốc độ cao, độ trễ thấp và khả năng quản lý linh hoạt, đã đặt ra những thách thức mới đối với hạ tầng mạng truyền thống. Kiến trúc mạng truy cập vô tuyến truyền thống (RAN) với các thiết bị và giao diện đóng độc quyền gây ra nhiều hạn chế về tính linh hoạt, khả năng tương tác và đổi mới công nghệ. Chính vì vậy, xu hướng chuyển dịch sang kiến trúc mạng truy cập vô tuyến mở (Open RAN) ngày càng thu hút sự quan tâm lớn từ cộng đồng nghiên cứu và các nhà cung cấp dịch vụ viễn thông hàng đầu thế giới.

Open RAN được kỳ vọng sẽ tái định hình ngành công nghiệp viễn thông nhờ vào việc mở các giao diện, phân tách chức năng mạng (như Central Unit - CU, Distributed Unit - DU, và Radio Unit - RU), đồng thời đưa trí tuệ nhân tạo (AI) vào sâu hơn trong việc quản lý và vận hành mạng lưới. Một thành phần cốt lõi giúp đạt được những lợi ích này chính là bộ điều khiển thông minh mạng vô tuyến (RAN Intelligent Controller - RIC) với hai phiên bản chính là Near-Real-Time RIC và Non-Real-Time RIC. Hai loại RIC này giúp quản lý và tối ưu các ứng dụng mạng (xApps và rApps) trong thời gian thực hoặc gần thực.

Mục tiêu của báo cáo này là làm rõ những ưu điểm của Open RAN so với mạng RAN truyền thống, đồng thời phân tích cụ thể vai trò của các thành phần quan trọng như RIC và các giao diện mở (E2, A1, O1). Qua đó, nhóm nghiên cứu mong muốn đề xuất một mô hình tối ưu hóa tài nguyên mạng hiệu quả bằng công nghệ học tăng cường sâu (DRL) nhằm đáp ứng các yêu cầu về hiệu năng, linh hoạt và khả năng mở rộng của mạng viễn thông thế hệ mới.

\chapter{Kiến thức cơ sở}
% \input{chapters/chapter02}

\chapter{Phương pháp tối ưu hoá ABR}
% \input{chapters/chapter03}

\chapter{Cài đặt phương pháp và thực nghiệm}
% Báo cáo đã trình bày chi tiết các vấn đề cốt lõi liên quan đến kiến trúc mạng truy cập vô tuyến mở (Open RAN), với việc tập trung vào vai trò của các thành phần quan trọng như RIC và các giao diện mở (E2, A1, O1). Thông qua nghiên cứu, rõ ràng thấy được các lợi ích nổi bật của Open RAN so với các kiến trúc mạng truyền thống, đặc biệt là khả năng linh hoạt, tương tác đa nhà cung cấp và triển khai các ứng dụng AI và DRL thông minh trong việc quản lý tài nguyên mạng.

Phần thực nghiệm đã được thực hiện bằng việc sử dụng bộ công cụ mô phỏng RIMEDO-TS, thể hiện rõ tính ứng dụng thực tiễn của kiến trúc Open RAN và công nghệ điều khiển thông minh DRL. Kết quả thực nghiệm chứng minh rằng việc triển khai các xApp thông minh dựa trên thuật toán DQN mang lại hiệu quả cao trong việc cân bằng tải, tối ưu hóa tài nguyên và nâng cao hiệu suất tổng thể của mạng.

Kết quả nghiên cứu và thực nghiệm khẳng định rằng kiến trúc Open RAN cùng với công nghệ trí tuệ nhân tạo và học tăng cường sâu có tiềm năng lớn trong việc tái định hình và nâng cấp mạng viễn thông hiện đại. Việc tiếp tục nghiên cứu sâu hơn và mở rộng các ứng dụng này trong tương lai sẽ giúp tối ưu hóa đáng kể hiệu năng và chi phí vận hành của các nhà cung cấp dịch vụ viễn thông, góp phần quan trọng vào sự phát triển của ngành công nghiệp viễn thông nói chung.



\chapter{Kết luận}
% Báo cáo đã trình bày chi tiết các vấn đề cốt lõi liên quan đến kiến trúc mạng truy cập vô tuyến mở (Open RAN), với việc tập trung vào vai trò của các thành phần quan trọng như RIC và các giao diện mở (E2, A1, O1). Thông qua nghiên cứu, rõ ràng thấy được các lợi ích nổi bật của Open RAN so với các kiến trúc mạng truyền thống, đặc biệt là khả năng linh hoạt, tương tác đa nhà cung cấp và triển khai các ứng dụng AI và DRL thông minh trong việc quản lý tài nguyên mạng.

Phần thực nghiệm đã được thực hiện bằng việc sử dụng bộ công cụ mô phỏng RIMEDO-TS, thể hiện rõ tính ứng dụng thực tiễn của kiến trúc Open RAN và công nghệ điều khiển thông minh DRL. Kết quả thực nghiệm chứng minh rằng việc triển khai các xApp thông minh dựa trên thuật toán DQN mang lại hiệu quả cao trong việc cân bằng tải, tối ưu hóa tài nguyên và nâng cao hiệu suất tổng thể của mạng.

Kết quả nghiên cứu và thực nghiệm khẳng định rằng kiến trúc Open RAN cùng với công nghệ trí tuệ nhân tạo và học tăng cường sâu có tiềm năng lớn trong việc tái định hình và nâng cấp mạng viễn thông hiện đại. Việc tiếp tục nghiên cứu sâu hơn và mở rộng các ứng dụng này trong tương lai sẽ giúp tối ưu hóa đáng kể hiệu năng và chi phí vận hành của các nhà cung cấp dịch vụ viễn thông, góp phần quan trọng vào sự phát triển của ngành công nghiệp viễn thông nói chung.



\chapter*{Tài liệu tham khảo}
% \noindent [1] Shaka Player. https://shaka-player-demo.appspot.com/docs/api/index.html. Accessed: May 15, 2025.

\noindent [2] Adam, Giuseppe Samela. (2023). Adaptive Bitrate Streaming Evolved: WISH ABR and Bitmovin’s Player Integration. https://bitmovin.com/blog/wish-abr-adaptive-bitrate-streaming. Accessed: May 16, 2025.

\noindent [3] Vijaya Sagar Vinnakota. (2021). A TL;DR on ABR algorithms in MPEG DASH.
https://www.linkedin.com/pulse/tldr-abr-algorithms-mpeg-dash-vijaya-sagar-vinnakota. Accessed: May 17, 2025.

\noindent [4] Ziyu Zhong, Mufan Liu, Le Yang, Yifan Wang, Yiling Xu, Jenq-Neng Hwang. (2025). Video Streaming with Kairos: An MPC-Based ABR with Streaming-Aware Throughput Prediction. https://cs.paperswithcode.com/paper/video-streaming-with-kairos-an-mpc-based-abr. Accessed: May 19, 2025.

\noindent [5] Yueheng Li, Qianyuan Zheng, Zicheng Zhang, Hao Chen, Zhan Ma. (2025). Improving ABR Performance for Short Video Streaming Using
Multi-Agent Reinforcement Learning with Expert Guidance. https://arxiv.org/pdf/2304.04637. Accessed: May 22, 2025.

\noindent [6] Huang, Tianchi and Yao, Xin and Wu, Chenglei and Zhang, Rui-Xiao and Sun, Lifeng. (2018). Tiyuntsong: A Self-Play Reinforcement Learning Approach for ABR Video Streaming. https://github.com/thu-media/Tiyuntsong. Accessed: May 23, 2025.

\noindent [7] Lisa. What is WISH ABR? Revolutionizing Adaptive Streaming Quality.
https://www.coconut.co/articles/wish-abr-revolutionizing-adaptive-streaming-quality. Accessed: Jun 3, 2025.

\noindent [8] Version History of Shaka Player. https://github.com/shaka-project/shaka-player/releases?utm\_source=chatgpt.com. Accessed: Jun 4, 2025.



% \printbibliography[heading=bibintoc]

\end{document}