\section{Mất gói tin điều khiển và ảnh hưởng tới quyết định hệ thống}
Có 4 loại gói control quan trọng trong giao thức: INI-U, INI-R, SET-U, SET-R (ngoài ra ACK nữa nhưng như bài báo lưu ý, ACK được giả định luôn truyền thành công nhờ bảo mật nội dung ngắn, ta bỏ qua). Xét từng trường hợp mất gói:

\begin{itemize}
    \item Mất gói INI-U (UE -> AP): Gói này mang thông tin trạng thái hàng đợi của UE gửi lên ES trước khi RA. Nếu mất, ES không biết lượng dữ liệu mới UE có. Bài báo giả định ES sẽ suy luận dựa trên dữ liệu đã nhận: nếu gói INI-U mất, ES tạm giả sử hàng đợi UE không có dữ liệu mới ngoài cái đã gửi ở slot trước . Điều này có thể sai nếu trong thời gian qua có thêm dữ liệu. Hệ quả: RA có thể cấp tài nguyên ít hơn cần thiết cho UE đó (vì tưởng UE gần hết data), dẫn đến trễ tăng do phần dữ liệu mới không được offload kịp. Tuy nhiên, vì RA còn tối ưu chung, UE này có thể bị thiệt thòi. Giải pháp: có thể thiết kế UE tự gửi lại INI-U nếu không nhận được phản hồi trong thời gian (nhưng phức tạp) hoặc dự đoán lượng đến.
    \item Mất gói INI-R (AP -> RIS): Gói này ra lệnh RIS chuyển sang chế độ ước lượng kênh (thay đổi cấu hình codebook). Nếu RIS không nhận được, nó sẽ không biết cần làm gì. Bài báo giả sử trường hợp xấu: RIS vẫn giữ cấu hình mặc định (ctl) trong suốt quá trình CE, nghĩa là AP chỉ nhận được pilot của kênh trực tiếp, không thu được gì từ kênh phản xạ . Về bản chất, CE phản xạ thất bại -> AP không có CSI về kênh qua RIS. Nếu không có chiến lược dự phòng, ES sẽ không thể tối ưu RIS (vì không biết kênh), có thể phải chọn config RIS ngẫu nhiên hoặc xấu nhất. Bài báo giả định trong trường hợp này, hệ thống không có CSI RIS nên đặt thông lượng danh định = 0 cho các UE qua RIS . Nói cách khác, coi như phiên offload thất bại hoàn toàn (vì không dám offload gì). Đây là tình huống nghiêm trọng nhất: mất 1 gói INI-R có thể làm lỡ toàn bộ slot cho nhiều UE. Do đó, gói INI-R cần được truyền thật tin cậy. Giải pháp: có thể dùng kênh out-of-band hoặc lặp lại gói; hoặc thiết kế nếu INI-R không ACK, AP \& RIS có mặc định an toàn (như luôn chạy quy trình beam sweeping thay vì tắt). Rõ ràng, kênh RIS-CC phải rất tin cậy để MEC/RIS vận hành tốt.
    \item Mất gói SET-U (AP -> UE): Gói này mang các tham số tối ưu (công suất, tốc độ) cho UE để UE truyền trong payload. Nếu UE không nhận được, nó không biết mình nên truyền thế nào. Bài báo giả định UE sẽ phải dùng lại cấu hình của slot trước (về công suất và tốc độ dữ liệu) và hy vọng AP giải mã được . Chiến lược này hợp lý: nếu kênh không thay đổi nhiều, dùng cấu hình cũ có thể vẫn ổn; nhưng nếu kênh đã thay đổi hoặc RA mới phân tài nguyên khác (ví dụ tăng rate), thì UE có thể đang dùng param lạc hậu. Trường hợp xấu: nếu lẽ ra RA muốn UE tăng công suất vì kênh xấu đi, mà UE vẫn phát yếu như slot trước, khả năng gói data này bị lỗi (do SNR không đủ cho MCS cũ). Hoặc ngược lại, RA giảm phân bổ cho UE (ví dụ ít băng thông hơn) mà UE vẫn gửi nhiều như trước -> tràn gói, xung đột. Bài báo đơn giản xét kịch bản: UE cứ phát như trước, AP cố giải mã, có thể lỗi tùy kênh  . Như vậy, mất SET-U thường ảnh hưởng cục bộ một UE, không làm hỏng toàn hệ nhưng gây giảm throughput và cần truyền lại. Giải pháp: AP có thể lắng nghe UL, nếu thấy UE dùng sai param có thể yêu cầu dừng (nhưng trong 1 slot ngắn khó kịp). Tốt hơn là nâng độ tin cậy UE-CC (mã hóa, lặp).
    \item Mất gói SET-R (AP -> RIS): Gói này mang cấu hình RIS tối ưu cho payload. Nếu RIS không nhận, nó sẽ không thể áp dụng cấu hình mới. Có thể RIS sẽ giữ nguyên cấu hình cũ từ slot trước (vì không có lệnh mới) hoặc tệ hơn, do ta đã chuyển sang control mode ở pha Setup, RIS có thể vẫn ở chế độ ctl rộng, không quay về cấu hình trước (tùy implement). Bài báo cho biết nếu SET-R mất, RIS chỉ có thể dùng lại $\Phi$ cũ (có thể hiểu là giữ cấu hình slot $t-1$ cho slot $t$). Nếu kênh không đổi nhiều, cấu hình cũ có thể vẫn ok, nhưng nếu UE di chuyển, config cũ có thể không tối ưu cho vị trí mới -> throughput giảm. Trường hợp xấu: nếu slot trước RIS để ctl config (giả sử slot trước bị lỗi CE), thì slot này cũng ctl config -> giống như không sử dụng RIS hiệu quả. Vậy mất SET-R cũng giống hệ thống bỏ lỡ lợi ích RIS trong slot hiện tại. Hậu quả có thể trên mọi UE nếu kênh direct xấu, throughput sụt. 
\end{itemize}


Tóm lại, trong các gói, INI-R và SET-R là quan trọng cho toàn hệ (vì ảnh hưởng chức năng RIS), SET-U quan trọng cho từng UE, INI-U tác động nhẹ nhất (ES có thể đoán). Fig.4 trong bài báo minh họa hiệu năng hệ thống theo xác suất lỗi gói điều khiển cho từng loại gói
ar5iv.org
. Kết quả chỉ ra, khi xác suất lỗi tăng, hiệu năng giảm mạnh, trong đó đường cong ứng với lỗi INI-R và SET-R suy giảm nặng nhất – khẳng định tầm quan trọng của chúng. Với xác suất lỗi control tầm $10^{-2}$ (1 phần trăm), độ trễ trung bình có thể tăng hàng chục phần trăm
. Điều này cho thấy MEC/RIS yêu cầu kênh điều khiển rất đáng tin cậy (tương đương URLLC).


Giải pháp tổng thể:
\begin{itemize}
    \item Dùng mã hóa kênh mạnh cho gói control (ví dụ mã LDPC dài, thấp tốc độ) để đảm bảo xác suất lỗi cực thấp, chấp nhận overhead tăng chút.
    \item Sử dụng công nghệ diversity: phát lặp lại gói điều khiển vài lần, hoặc gửi qua nhiều tần số/tuyến đường (nếu có).
    \item Thiết kế giao thức dự phòng: nếu phát hiện mất gói (ví dụ UE không nhận SET-U có thể báo NACK, AP re-send nhanh trong slot), hoặc fallback như trong trường hợp UE dùng cấu hình cũ, RIS giữ config cũ.
    \item Sử dụng kênh out-of-band cho RIS-CC, UE-CC trong ứng dụng đòi hỏi nghiêm ngặt, để tách biệt hẳn khỏi data, có thể điều chế robust hơn.
\end{itemize}