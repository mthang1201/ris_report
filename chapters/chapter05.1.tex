\section{Hiệu năng hệ thống theo tỷ lệ overhead điều khiển}

Hiệu năng thông lượng vs. overhead: Nếu bỏ qua lỗi, khi tăng overhead (thêm thời gian cho CE, RA), hệ thống đạt cấu hình gần tối ưu hơn, do đó thông lượng dữ liệu mỗi slot tăng (nhờ CSI chính xác, RIS tối ưu). Tuy nhiên, do thời gian truyền dữ liệu bị rút ngắn, nên lượng dữ liệu thực sự truyền được có thể không tăng tương ứng. Có điểm tới hạn: khi overhead vượt quá mức, thông lượng net giảm. Mô phỏng đã khẳng định có một tỷ lệ overhead tối ưu (khoảng 10-20pt tùy kịch bản) để tối đa hóa throughput net. Hiệu năng năng lượng vs. overhead: Một mặt, overhead cao giúp giảm năng lượng phát (vì cấu hình tốt, ít lãng phí công suất), nhưng mặt khác overhead chính nó tiêu tốn năng lượng (pilot dài, tính toán nhiều). Tổng năng lượng (UE+mạng) thường có điểm tối ưu tại overhead vừa phải. Kết quả từ Fig.3 (error-free) cho thấy năng lượng trung bình hệ thống thấp nhất tại một độ dài slot nhất định (tương ứng overhead ~15).


Khi overhead tăng hơn mức này, lợi ích giảm nhiêu liệu (vì UE có thể giảm công suất) không đủ bù đắp năng lượng phần điều khiển. Hiệu năng độ trễ vs. overhead: Overhead quá ít có thể gây thông lượng thấp do cấu hình xấu, khi đó hàng đợi kéo dài, tăng trễ. Overhead quá nhiều thì trực tiếp giảm thời gian phục vụ dữ liệu, trễ cũng tăng. Do đó, về độ trễ, cũng có mức overhead tối ưu. Thông thường, mức overhead tối ưu về trễ tương đồng với tối ưu năng lượng, vì trễ thấp thường đạt được khi hiệu quả năng lượng tốt (không lãng phí thời gian, không backlog). Một khía cạnh nữa: overhead tối ưu còn phụ thuộc mục tiêu QoS cụ thể. Nếu mục tiêu là độ trễ cực thấp (URLLC), ta có thể chấp nhận tốn năng lượng hơn miễn sao trễ giảm – tức sẵn sàng overhead nhiều để chắc chắn không mất gói, không phải truyền lại. Ngược lại nếu mục tiêu tiết kiệm năng lượng tối đa, có thể chịu trễ nhỉnh hơn chút, overhead có thể giảm.