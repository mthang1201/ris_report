\section{Giao diện mở (E2, A1, O1) và vai trò của chúng}

Giao diện E2 là giao diện mở kết nối trực tiếp bộ điều khiển RAN thông minh thời gian thực gần (Near-RT RIC) với các node mạng RAN (gọi chung là E2-nodes), bao gồm các thành phần như O-CU (khối xử lý tập trung – control/user plane), O-DU (khối xử lý phân tán) và thậm chí cả eNB/gNB nếu tuân thủ O-RAN. Mục đích thiết kế của E2 là cho phép RIC thời gian thực gần thu thập số liệu và trạng thái từ mạng RAN theo thời gian thực, đồng thời gửi các lệnh điều khiển tương ứng xuống các nút mạng này một cách nhanh chóng. Giao diện E2 tạo kênh trao đổi hai chiều: truyền lên các thông tin đo lường, chỉ số hiệu năng, sự kiện (telemetry) từ RAN lên Near-RT RIC, và truyền xuống các lệnh cấu hình, điều khiển từ RIC tới RAN \cite{frontiers}. Với E2, Near-RT RIC (thông qua các xApp chạy trên đó) có thể thực hiện các vòng lặp điều khiển kín gần thời gian thực cho RAN. Cụ thể, E2 hỗ trợ các dịch vụ như giám sát (monitor), tạm dừng hoặc ghi đè (suspend/override) và điều khiển các chức năng trên nút mạng, giúp RIC có thể can thiệp vào hoạt động nội tại của trạm gốc khi cần thiết \cite{frontiers}. Các xApp – là những ứng dụng tác nghiệp thời gian thực gần trên RIC – sử dụng E2 để đăng ký nhận các chỉ số/KPI từ nút mạng và đưa ra quyết định điều khiển tương ứng (ví dụ: điều chỉnh lịch truyền, phân bổ tài nguyên vô tuyến, cân bằng tải, giảm nhiễu) trong khoảng thời gian tính bằng mili-giây đến dưới 1 giây. Nhờ E2, O-RAN cho phép tách rời và mở rộng khả năng quản lý tài nguyên vô tuyến của trạm gốc, tạo điều kiện cho nhiều nhà cung cấp khác nhau tích hợp giải pháp tối ưu riêng vào RAN một cách tiêu chuẩn hóa \cite{frontiers}


Giao diện A1 được định nghĩa để kết nối RIC phi thời gian thực (Non-RT RIC, tích hợp trong khung SMO) với Near-RT RIC. Đây là kênh trao đổi thông tin ở mức chiến lược và chính sách giữa lớp quản lý/tối ưu dài hạn và lớp điều khiển ngắn hạn của RAN. Mục đích thiết kế của A1 là cho phép Non-RT RIC cung cấp định hướng vận hành cho Near-RT RIC dưới dạng các chính sách tối ưu mạng (policy) hoặc thông tin hỗ trợ thông minh. Chẳng hạn, thông qua A1, rApp (ứng dụng chạy trên Non-RT RIC) có thể gửi xuống Near-RT RIC những chính sách điều khiển (ví dụ: ưu tiên phục vụ cho một lát cắt mạng hoặc hạn mức tài nguyên cho một nhóm UE cụ thể) nhằm định hướng hành vi của các xApp trên RIC thời gian thực gần \cite{frontiers}. Chức năng chính của A1 bao gồm truyền tải policy (chính sách điều khiển) và thông tin làm giàu (enrichment information) từ khối Non-RT RIC xuống Near-RT RIC. Các policy A1 được định nghĩa ở mức ý định/cấp cao, ví dụ như mục tiêu QoS hoặc KPI mà hệ thống cần đạt được cho một tập người dùng hoặc một lát cắt mạng \cite{Understanding_O-Ran}. Giao diện A1 cũng được thiết kế để hỗ trợ việc quản lý vòng đời của các mô hình học máy (ML) được sử dụng trong RAN, chẳng hạn như phân phối hoặc cập nhật mô hình ML cho xApp, mặc dù chức năng này vẫn đang được nghiên cứu bổ sung. Ngoài ra, A1 cho phép Non-RT RIC gửi thông tin làm giàu – ví dụ dữ liệu dự đoán từ phân tích dài hạn hoặc thông tin từ nguồn ngoại vi – để hỗ trợ xApp ra quyết định tốt hơn. Ngược lại, Near-RT RIC có thể phản hồi tối thiểu qua A1 về trạng thái thực thi chính sách (ví dụ chính sách đã được áp dụng hay chưa) để Non-RT RIC theo dõi \cite{Understanding_O-Ran}. Với vai trò cầu nối ở mức phi thời gian thực, A1 đảm bảo rằng những tối ưu dài hạn (từ rApp) được hiện thực hóa kịp thời trong lớp điều khiển RAN ngắn hạn, tạo nên sự kết hợp nhịp nhàng giữa chiến lược tổng thể và tác vụ điều khiển cụ thể trong hệ thống RAN thông minh.


Giao diện O1 là giao diện mở hỗ trợ chức năng quản lý, vận hành và bảo trì (OAM) cho tất cả các thành phần O-RAN, nằm trong khung Quản lý dịch vụ và điều phối (SMO) của hệ thống. Cụ thể, O1 kết nối hệ thống SMO với các phần tử mạng O-RAN (O-CU, O-DU, O-RU, Near-RT RIC, v.v.), cho phép thực hiện các tác vụ FCAPS truyền thống – bao gồm Quản lý lỗi, Cấu hình, Kế toán, Hiệu năng và Bảo mật – trên môi trường RAN mở đa nhà cung cấp \cite{frontiers}. Mục đích thiết kế của giao diện O1 là cung cấp một kênh quản lý tập trung và tiêu chuẩn để cấu hình tham số mạng, phân bổ tài nguyên, giám sát hiệu năng và thu thập dữ liệu thống kê từ các nút mạng khác nhau, bất kể nhà sản xuất. Chức năng chính của O1 bao gồm: quản lý cấu hình (ví dụ: thiết lập cấu hình ban đầu cho O-CU/O-DU/O-RU, điều chỉnh tham số RAN theo lịch hoặc theo yêu cầu tối ưu), giám sát hiệu năng (thu thập KPI, số liệu thống kê, bản ghi sự kiện từ các node mạng), quản lý lỗi (nhận cảnh báo, nhật ký lỗi từ phần tử mạng) và quản lý phần mềm (nâng cấp phần mềm, thay đổi phiên bản cấu kiện mạng) \cite{frontiers}. Trong kiến trúc RAN thông minh, O1 cung cấp dữ liệu nền tảng cho khối Non-RT RIC/rApps phân tích. Chẳng hạn, rApp có thể sử dụng số liệu thu thập qua O1 (như mức tải, công suất, chất lượng kênh từ các cell) để phát hiện xu hướng hoặc bất thường, từ đó đề xuất chính sách tối ưu tương ứng. Đồng thời, O1 cũng thực thi các quyết định quản trị như điều chỉnh cấu hình mạng hoặc bật/tắt tài nguyên vô tuyến khi được yêu cầu (thường là kết quả của các thuật toán tối ưu dài hạn). Nhờ giao diện O1, việc vận hành RAN trở nên linh hoạt và tự động hơn, tạo nền tảng cho quản trị mạng tự động trong môi trường O-RAN. 


\textbf{So sánh và tương tác}: Ba giao diện E2, A1, O1 đảm nhiệm các vai trò bổ trợ nhau ở các tầng thời gian khác nhau trong hệ thống RAN thông minh \cite{Understanding_O-Ran}. Thứ nhất, E2 là giao diện thời gian thực gần (độ trễ khoảng 10 ms đến dưới 1 giây) phục vụ cho vòng điều khiển nhanh giữa Near-RT RIC và các nút mạng RAN. Giao diện này cho phép các xApp thu thập tức thời trạng thái mạng và tác động trực tiếp lên tài nguyên vô tuyến (ví dụ điều khiển lịch truyền, phân bổ kênh) nhằm tối ưu cục bộ theo thời gian thực. Thứ hai, A1 hoạt động ở mức phi thời gian thực (độ trễ từ vài giây trở lên), là kênh truyền tải chính sách và thông tin thông minh từ tầng quản lý xuống tầng điều khiển \cite{frontiers}. A1 đảm bảo Near-RT RIC vận hành theo đúng mục tiêu tối ưu dài hạn do Non-RT RIC đề ra (ví dụ đảm bảo thông lượng cell edge, giảm tiêu thụ năng lượng...), hỗ trợ bởi các mô hình ML và phân tích dữ liệu ở tầng trên. Thứ ba, O1 là giao diện quản lý với thời gian phản hồi không đòi hỏi tức thời, chủ yếu phục vụ công tác OAM và cung cấp dữ liệu cho các thuật toán thông minh \cite{Understanding_O-Ran}. O1 cho phép SMO và rApp nhìn bao quát toàn mạng, thu thập số liệu đa miền (radio, truyền dẫn, core) để từ đó tối ưu cấu hình mạng một cách tự động. Trong một chu trình vận hành RAN thông minh, cả ba giao diện kết hợp nhịp nhàng theo mô hình điều khiển phân tầng. Non-RT RIC sử dụng O1 để thu thập số liệu vận hành từ các node RAN (đồng thời nhận thông tin từ core mạng nếu cần), tiến hành phân tích và huấn luyện mô hình ML. Từ kết quả phân tích, rApp sinh ra các chính sách hoặc khuyến nghị tối ưu và gửi xuống Near-RT RIC qua giao diện A1\cite{Understanding_O-Ran}. Near-RT RIC nhận được chỉ dẫn từ A1 sẽ điều chỉnh hành vi của các xApp, cho phép các xApp tác động lên mạng RAN theo thời gian thực gần thông qua giao diện E2 (ví dụ thay đổi tham số lịch trình, hạn chế kết nối vào một cell quá tải, v.v.). Như vậy, E2, A1, O1 tạo thành bộ khung giao diện mở giúp hiện thực hóa RAN mở và thông minh: O1 cung cấp khả năng quan sát và cấu hình toàn diện, A1 truyền tải tri thức và mục tiêu tối ưu, còn E2 thực thi điều khiển linh hoạt tại nút mạng, tất cả đều trên nền tảng tiêu chuẩn chung của O-RAN \cite{frontiers}. Nguồn tài liệu tham khảo: Các thông tin trên được tổng hợp từ đặc tả kỹ thuật của O-RAN Alliance và các nghiên cứu IEEE gần đây. Đặc biệt, định nghĩa và chức năng của các giao diện E2, A1, O1 được mô tả trong tài liệu của O-RAN Alliance \cite{frontiers}, cũng như các phân tích học thuật về kiến trúc O-RAN (như bài báo của Michele Polese và cộng sự) để làm rõ vai trò của chúng trong hệ thống RAN thông minh. Các ví dụ triển khai và use-case minh họa cũng cho thấy sự phối hợp giữa ba giao diện này trong việc hỗ trợ ứng dụng rApp/xApp tối ưu mạng một cách linh hoạt và hiệu quả.
