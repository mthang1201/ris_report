\section{Ứng dụng RIS trong MEC để giảm độ trễ}

Việc kết hợp RIS vào hệ thống MEC (tức RIS-aided MEC) được kỳ vọng nâng cao hiệu năng hệ thống trên nhiều phương diện. Thứ nhất, RIS giúp cải thiện chất lượng kênh truyền giữa UE và AP, từ đó tăng tốc độ offload dữ liệu từ UE lên edge server (ES)
\cite{ris_latency}
. Tốc độ offload cao cho phép UE gửi xong dữ liệu sớm hơn, ES xử lý sớm và trả kết quả sớm, nên giảm độ trễ tổng cho ứng dụng. Thứ hai, với sự hỗ trợ của RIS, MEC có thể phục vụ các UE ở vị trí trước đây sóng khó tới (góc khuất, vùng biên cell) mà vẫn đảm bảo băng thông, giúp mở rộng vùng phục vụ MEC mà không cần tăng công suất phát hay triển khai thêm trạm. Thứ ba, RIS có thể tiết kiệm năng lượng hệ thống: do kênh được cải thiện, UE có thể truyền ở công suất thấp hơn cho cùng lượng dữ liệu, hoặc ES không cần chờ nhiều lần truyền lại do lỗi, qua đó tổng năng lượng dùng để hoàn thành tác vụ giảm đi. 



Một ví dụ điển hình, công trình của P. Di Lorenzo và cs. (2022) đã nghiên cứu MEC động có RIS, trong đó UE liên tục sinh dữ liệu cần xử lý và kênh vô tuyến thay đổi theo thời gian. Kết quả cho thấy với RIS, có thể tối ưu hóa phối hợp cấu hình RIS, tham số truyền thông (phân bổ băng thông, công suất) và tài nguyên tính toán tại ES để đảm bảo độ trễ trung bình dưới ngưỡng trong khi tối thiểu hóa tiêu thụ năng lượng của toàn hệ (cả UE lẫn mạng)
\cite{mec}
. Tuy nhiên, tất cả các tối ưu kể trên giả định lý tưởng rằng thông tin kênh qua RIS và trạng thái hệ thống đều biết hoàn hảo và ngay lập tức. Như đã nêu, thực tế để đạt được điều đó cần các thủ tục điều khiển phức tạp. Phần tiếp theo, chúng tôi đi vào chi tiết mô hình hệ thống và các thủ tục này.
3. Mô hình hệ thống MEC h