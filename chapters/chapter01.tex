Mạng không dây thế hệ sau 5G và hướng tới 6G dự kiến kích hoạt nhiều dịch vụ và ứng dụng mới (ví dụ: Công nghiệp 4.0, Internet vạn vật - IoT, xe tự hành), vốn đòi hỏi xử lý dữ liệu lớn với độ tin cậy cao và độ trễ đầu-cuối thấp. Để đáp ứng các yêu cầu này, tính toán biên di động (MEC) đã nổi lên như một công nghệ chủ chốt, cho phép đưa khả năng điện toán và lưu trữ đám mây đến gần biên mạng (ví dụ: tại các trạm truy cập không dây). Thay vì xử lý toàn bộ tác vụ trên thiết bị người dùng (UE) hoặc gửi lên đám mây trung tâm (với độ trễ cao), MEC cho phép offload các tác vụ nặng tới máy chủ biên gần UE để giảm độ trễ truyền và tiết kiệm năng lượng cho thiết bị. Nhiều nghiên cứu đã đề xuất các giải pháp tối ưu tài nguyên mạng cho MEC nhằm đảm bảo chất lượng dịch vụ (QoS) thỏa đáng cho người dùng, trong cả kịch bản tĩnh và động. 


Tuy nhiên, hiệu năng của hệ thống MEC phụ thuộc rất lớn vào chất lượng kết nối không dây giữa UE và điểm truy cập (AP) tích hợp máy chủ biên. Nếu kênh vô tuyến xấu (suy hao, chặn, nhiễu cao), tốc độ offload giảm mạnh, dẫn đến hàng đợi tác vụ ùn ứ và trễ xử lý tăng lên, làm mất lợi thế của MEC. Gần đây, công nghệ Reconfigurable Intelligent Surface (RIS) được nghiên cứu nhiều nhằm điều khiển môi trường truyền sóng không dây, cải thiện chất lượng kênh vật lý. RIS gồm một mảng bề mặt nhiều phần tử có thể điều chỉnh pha (và biên độ) phản xạ sóng điện từ, qua đó hướng tín hiệu đến nơi mong muốn và tăng cường chất lượng liên kết mà không cần truyền thêm công suất phát
. Việc tích hợp RIS trong hệ thống MEC hứa hẹn nâng cao tốc độ offload và mở rộng vùng phủ sóng hiệu quả của MEC, ngay cả khi kênh truyền thẳng UE-AP bị chặn hay suy hao nặng. Các nghiên cứu gần đây đã xem xét tối ưu hóa MEC có RIS (RIS-aided MEC) trong nhiều kịch bản: từ offload tĩnh đến động, từ tối ưu hóa truyền thống đến sử dụng học máy.



Tuy nhiên, một khoảng trống trong các nghiên cứu MEC/RIS trước đây là chưa xem xét đầy đủ các thủ tục điều khiển cần thiết để vận hành RIS trong hệ thống MEC cũng như tác động của chúng đến QoS người dùng cuối. Hầu hết các công trình giả định lý tưởng rằng RIS có thể điều chỉnh tức thì và hệ thống luôn có thông tin kênh hoàn hảo mà không tính đến chi phí thu thập thông tin đó. Trên thực tế, để sử dụng được RIS, hệ thống phải thực hiện một loạt thao tác điều khiển như ước lượng kênh (Channel Estimation - CE) cho các liên kết qua RIS, tối ưu phân bổ tài nguyên (Resource Allocation - RA) gồm lựa chọn cấu hình RIS, điều chỉnh công suất, băng thông truyền,..., và trao đổi tín hiệu điều khiển giữa các thực thể (AP, RIS, UE, máy chủ) để phối hợp vận hành. Những tác vụ điều khiển này tiêu tốn thời gian và tài nguyên của hệ thống – gọi chung là overhead điều khiển – và do đó ảnh hưởng trực tiếp đến độ trễ người dùng cảm nhận cũng như tiêu thụ năng lượng của mạng.