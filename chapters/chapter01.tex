Trong những năm gần đây, sự phát triển bùng nổ của các dịch vụ viễn thông, đặc biệt là các ứng dụng yêu cầu tốc độ cao, độ trễ thấp và khả năng quản lý linh hoạt, đã đặt ra những thách thức mới đối với hạ tầng mạng truyền thống. Kiến trúc mạng truy cập vô tuyến truyền thống (RAN) với các thiết bị và giao diện đóng độc quyền gây ra nhiều hạn chế về tính linh hoạt, khả năng tương tác và đổi mới công nghệ. Chính vì vậy, xu hướng chuyển dịch sang kiến trúc mạng truy cập vô tuyến mở (Open RAN) ngày càng thu hút sự quan tâm lớn từ cộng đồng nghiên cứu và các nhà cung cấp dịch vụ viễn thông hàng đầu thế giới.

Open RAN được kỳ vọng sẽ tái định hình ngành công nghiệp viễn thông nhờ vào việc mở các giao diện, phân tách chức năng mạng (như Central Unit - CU, Distributed Unit - DU, và Radio Unit - RU), đồng thời đưa trí tuệ nhân tạo (AI) vào sâu hơn trong việc quản lý và vận hành mạng lưới. Một thành phần cốt lõi giúp đạt được những lợi ích này chính là bộ điều khiển thông minh mạng vô tuyến (RAN Intelligent Controller - RIC) với hai phiên bản chính là Near-Real-Time RIC và Non-Real-Time RIC. Hai loại RIC này giúp quản lý và tối ưu các ứng dụng mạng (xApps và rApps) trong thời gian thực hoặc gần thực.

Mục tiêu của báo cáo này là làm rõ những ưu điểm của Open RAN so với mạng RAN truyền thống, đồng thời phân tích cụ thể vai trò của các thành phần quan trọng như RIC và các giao diện mở (E2, A1, O1). Qua đó, nhóm nghiên cứu mong muốn đề xuất một mô hình tối ưu hóa tài nguyên mạng hiệu quả bằng công nghệ học tăng cường sâu (DRL) nhằm đáp ứng các yêu cầu về hiệu năng, linh hoạt và khả năng mở rộng của mạng viễn thông thế hệ mới.