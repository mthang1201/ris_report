\section{Chức năng của Near-Real-Time RIC và Non-Real-Time RIC}
Trong kiến trúc mạng truy cập vô tuyến mở (Open RAN), Bộ điều khiển RAN thông minh (RAN Intelligent Controller – RIC) được phân tách thành hai phần: RIC thời gian gần thực (Near-Real-Time RIC) và RIC thời gian không thực (Non-Real-Time RIC). Hai thành phần này phối hợp điều khiển và tối ưu mạng ở những quy mô thời gian khác nhau nhằm nâng cao hiệu năng của RAN. Cụ thể, Near-RT RIC đảm nhiệm việc điều khiển RAN với độ trễ thấp (từ khoảng 10 mili-giây đến <1 giây) \cite{Understanding_O-Ran}, còn Non-RT RIC phụ trách các tác vụ ở quy mô thời gian dài hơn (>1 giây, thường tính bằng giây, phút hoặc lâu hơn) \cite{Understanding_O-Ran}. Sự phân chia này cho phép tối ưu mạng ở cả thời gian thực ngắn hạn lẫn hoạch định dài hạn, tạo nên hệ thống điều khiển nhiều tầng cho RAN.

Near-Real-Time RIC (Near-RT RIC): Đây là thành phần RIC hoạt động gần thời gian thực, thường được triển khai trên hạ tầng điện toán biên hoặc cụm mạng khu vực gần với các nút RAN. Near-RT RIC có chức năng thu thập thông tin trạng thái mạng và thực thi các hành động điều khiển nhanh lên mạng vô tuyến với độ trễ yêu cầu dưới ~1 giây \cite{Understanding_O-Ran}. Theo đặc tả O-RAN, Near-RT RIC là một chức năng logic cho phép điều khiển và tối ưu tài nguyên RAN ở mức độ nhanh, thông qua việc thu thập dữ liệu chi tiết và tác động hành động lên các nút RAN qua giao diện E2 docs.o-ran-sc.org. Near-RT RIC thường xử lý các tác vụ như điều khiển truy cập vô tuyến và tài nguyên vô tuyến theo thời gian thực gần, ví dụ: điều phối lịch truyền dẫn, cân bằng tải giữa các cell, điều chỉnh tham số handover, điều khiển can nhiễu,… nhằm tối ưu hiệu suất thông lượng và chất lượng dịch vụ tức thời cho người dùng. Thành phần này tương tác trực tiếp với các nút mạng RAN (như O-DU, O-CU) qua giao diện E2 để nhận số liệu tình trạng (telemetry) và gửi chỉ thị điều khiển một cách liên tục. Đặc điểm quan trọng của Near-RT RIC là khả năng mở rộng chức năng qua các xApp – những ứng dụng plug-and-play chạy trên nền tảng Near-RT RIC để thực hiện các thuật toán điều khiển radio chuyên biệt \cite{Understanding_O-Ran}. Near-RT RIC cũng có cơ chế phối hợp và tránh xung đột giữa nhiều xApp khác nhau cùng tác động lên RAN (ví dụ: cơ chế quản lý message bus, lớp dữ liệu chia sẻ, và logic phân giải xung đột) để đảm bảo các quyết định điều khiển không mâu thuẫn \cite{Understanding_O-Ran}.

Non-Real-Time RIC (Non-RT RIC): Đây là thành phần RIC hoạt động ngoài thời gian thực chặt chẽ, nằm trong khối dịch vụ quản lý và điều hành (Service Management and Orchestration – SMO) ở trung tâm mạng hoặc đám mây. Non-RT RIC chịu trách nhiệm thực hiện các tác vụ quản lý, tối ưu RAN ở quy mô dài hạn hơn (trên 1 giây) \cite{Understanding_O-Ran}, bao gồm quản lý chính sách dịch vụ, phân tích hiệu năng, tối ưu cấu hình và hoạch định tài nguyên chiến lược cho mạng. Theo O-RAN Alliance, Non-RT RIC là một chức năng logic trong SMO hỗ trợ điều khiển/tối ưu RAN phi-thời-gian-thực, cung cấp khung AI/ML để huấn luyện và cập nhật mô hình, và truyền tải các hướng dẫn chính sách tới RIC gần thực \cite{Oran_overview} \cite{etsi_oranArchitecture}. Non-RT RIC được cấu thành bởi framework Non-RT RIC (nền tảng) và các ứng dụng rApp (các ứng dụng chạy trên Non-RT RIC). Nền tảng Non-RT RIC thực hiện việc kết thúc (terminate) giao diện A1 với Near-RT RIC, đồng thời phơi bày dịch vụ quản lý dữ liệu và ML cho các rApp thông qua giao diện nội bộ R1 \cite{etsi_oranArchitecture}. Các rApp (RAN applications) là những ứng dụng mô-đun chạy trên Non-RT RIC, sử dụng các dịch vụ mà nền tảng cung cấp để tạo ra các giá trị gia tăng cho vận hành RAN \cite{etsi_oranArchitecture}. Nhiệm vụ của rApp rất đa dạng, bao gồm: đề xuất và điều chỉnh chính sách điều khiển RAN, phân tích dữ liệu hiệu năng dài hạn, tối ưu cấu hình tham số, cũng như cung cấp thông tin bổ sung (enrichment information) cho các ứng dụng khác \cite{etsi_oranArchitecture}. Non-RT RIC gửi hướng dẫn chính sách và mục tiêu đến Near-RT RIC thông qua giao diện A1 (ví dụ: chính sách về phân bổ tài nguyên, mục tiêu QoS cần đạt, tham số ngưỡng sự kiện, v.v.), nhờ đó ảnh hưởng gián tiếp đến hành vi của các xApp trên Near-RT RIC \cite{etsi_oranArchitecture}. Ngược lại, Non-RT RIC cũng thu thập phản hồi từ mạng (thông qua dữ liệu O1 hoặc qua báo cáo từ Near-RT RIC) để đánh giá và điều chỉnh các chiến lược tối ưu. Có thể xem Non-RT RIC như “bộ não” ở tầng trên, vạch ra chiến lược dài hạn cho mạng, trong khi Near-RT RIC là “cánh tay tác động” ở tầng dưới thực thi các điều chỉnh nhanh theo chiến lược đó.