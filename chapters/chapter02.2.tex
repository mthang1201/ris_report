\section{Reconfigurable Intelligent Surface (RIS) – Bề mặt thông minh tái cấu hình}

RIS (còn gọi là Intelligent Reflecting Surface - IRS) là một công nghệ anten thông minh thụ động bao gồm hàng trăm tới hàng nghìn phần tử phản xạ có thể điều chỉnh được. Mỗi phần tử RIS có khả năng thay đổi pha (và biên độ) của sóng điện từ tới nó, thông qua đó có thể hướng các tia phản xạ tập trung về đích mong muốn, tương tự hiệu ứng beamforming nhưng thông qua phản xạ thay vì phát chủ động \cite{mec}. Điểm đặc biệt là RIS không có mạch phát sóng vô tuyến riêng (nếu là RIS thụ động) nên tiêu thụ năng lượng rất thấp; nó hoạt động như một “tấm gương thông minh” điều khiển hướng phản xạ bằng cách thiết lập trạng thái (pha) cho từng phần tử. 


Trong mạng không dây, RIS được coi là một thành phần hạ tầng mới có thể tái cấu hình môi trường truyền để cải thiện chất lượng liên kết. Ví dụ, nếu đường truyền trực tiếp UE–AP bị chắn, ta có thể đặt một RIS trên tường tòa nhà để nhận tín hiệu từ UE và phản xạ hướng về AP, tạo một đường truyền gián tiếp mạnh mẽ hơn (tín hiệu NLOS được bù suy hao nhờ RIS) \cite{ris_latency}. Nghiên cứu cho thấy RIS có thể tăng cường cả hiệu quả phổ và hiệu quả năng lượng của mạng không dây. RIS đặc biệt hấp dẫn cho 6G vì tính thụ động, chi phí thấp và có thể dễ dàng tích hợp (ốp vào tường, trần nhà, pano...). Nhiều ứng dụng RIS được đề xuất: tăng cường vùng phủ sóng trong nhà, tạo kênh truyền có lợi cho IoT công suất thấp, giảm nhiễu xuyên kênh trong mạng cell nhỏ, v.v.


Tuy nhiên, một hạn chế lớn của RIS thụ động là hiệu ứng “suy hao hai lần” (doubly path loss): do tín hiệu phải truyền hai chặng (UE–RIS và RIS–AP), tổng suy hao đường truyền nhân lên đáng kể so với truyền thẳng
\cite{mec}
. Để bù đắp, người ta có thể tăng số lượng phần tử RIS nhằm tích lũy độ lợi phản xạ thụ động cao hơn, nhưng điều này lại tăng overhead cho việc điều khiển RIS (phải ước lượng nhiều kênh và gửi nhiều thông tin điều khiển hơn)
. Gần đây, khái niệm RIS chủ động được đề xuất: các phần tử RIS được gắn thêm mạch khuếch đại tín hiệu, cho phép khuếch đại sóng phản xạ thay vì chỉ phản xạ thụ động, qua đó bù đắp suy hao hai chặng
. RIS chủ động có thể tăng SNR đáng kể và mở rộng tầm phủ sóng, nhưng đánh đổi bằng tiêu thụ năng lượng lớn hơn nhiều (mỗi phần tử như một bộ khuếch đại nhỏ) và có thể thêm nhiễu nhiệt do mạch khuếch đại gây ra
. Một câu hỏi đặt ra: RIS chủ động hay thụ động tốt hơn? Thực tế, mỗi loại có ưu nhược riêng và có tính bổ sung cho nhau
. Trong một số kịch bản, RIS chủ động cho SNR cao hơn hẳn; nhưng nếu năng lượng bị giới hạn hoặc nhiễu do khuếch đại lấn át, RIS thụ động có thể cho kết quả ổn định hơn. Các nghiên cứu gần đây đề xuất hệ lai chủ động-thụ động hoặc linh hoạt chuyển chế độ để tận dụng lợi ích của cả hai