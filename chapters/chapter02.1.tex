\section{Mobile Edge Computing (MEC) – Tính toán biên di động}


MEC là một kiến trúc tính toán phân tán, trong đó tài nguyên xử lý và lưu trữ đám mây được bố trí tại biên mạng di động, gần với phía người dùng hơn so với đám mây truyền thống. Ý tưởng chính của MEC là giảm khoảng cách vật lý giữa thiết bị người dùng (UE) và máy chủ thực thi tác vụ, qua đó giảm độ trễ truyền dẫn và tiết kiệm băng thông backhaul \cite{ris_latency}. Trong MEC, các UE có thể offload (tải lên) một phần hoặc toàn bộ tác vụ tính toán nặng (ví dụ: xử lý ảnh, video, phân tích dữ liệu IoT, AI...) đến máy chủ MEC đặt tại các trạm gốc hoặc điểm truy cập WiFi cục bộ. Máy chủ MEC, tuy có năng lực hạn chế hơn cloud trung tâm, nhưng nhờ vị trí gần UE nên có thể trả kết quả nhanh, đáp ứng yêu cầu độ trễ nghiêm ngặt của ứng dụng thời gian thực. 


Có hai chế độ offload phổ biến: (i) Offload toàn phần (binary offloading) – toàn bộ tác vụ được gửi lên MEC hoặc xử lý cục bộ, phù hợp với tác vụ nhỏ, không thể tách; (ii) Offload từng phần (partial offloading) – tác vụ được chia thành nhiều phần, một phần xử lý tại UE, phần còn lại gửi lên MEC, phù hợp với tác vụ lớn có thể song song hóa \cite{ris_latency}. MEC đã chứng minh hiệu quả trong giảm trễ so với điện toán đám mây truyền thống, nhưng vẫn gặp thách thức khi kênh vô tuyến không đảm bảo hoặc số lượng thiết bị lớn. Để nâng cao thông lượng offload, nhiều giải pháp bổ trợ đã được nghiên cứu: mạng di động không đồng nhất (HetNet) với các small-cell để giảm khoảng cách UE-AP; sử dụng massive MIMO tại trạm gốc để tăng phân tập và chống nhiễu; truyền ở băng tần mmWave hay THz để có băng thông rộng hơn cho offload; dùng UAV làm trạm di động để tạo đường truyền LOS linh hoạt. Tuy nhiên, các giải pháp này cũng có nhược điểm riêng (chi phí triển khai cao, kiến trúc phức tạp, tiêu thụ năng lượng lớn). Do đó, xuất hiện nhu cầu tìm kiếm giải pháp khác bổ trợ MEC hiệu quả, đặc biệt trong bối cảnh 6G.