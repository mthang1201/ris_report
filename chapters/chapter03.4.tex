\section{Kết quả và đánh giá thực nghiệm}

Sau khi triển khai và chạy mô phỏng, nhóm thu thập các chỉ số đánh giá hiệu năng của hệ thống mạng như:

\begin{itemize}
    \item Thông lượng mạng trung bình của người dùng.
    \item Tỷ lệ tải cân bằng giữa các gNB.
    \item Tỷ lệ chuyển giao (handover) thành công.
    \item Hiệu suất thuật toán DRL (độ hội tụ, reward trung bình mỗi episode).
\end{itemize}


Các kết quả mô phỏng ban đầu cho thấy rõ hiệu quả của việc sử dụng kiến trúc Open RAN kết hợp với các thuật toán DRL thông minh trong việc cải thiện hiệu năng hệ thống mạng vô tuyến, khẳng định tính khả thi và hiệu quả của phương pháp này trong việc quản lý tài nguyên và cân bằng tải trong mạng viễn thông tương lai.